
% Default to the notebook output style

    


% Inherit from the specified cell style.




    
\documentclass[11pt]{article}

    
    
    \usepackage[T1]{fontenc}
    % Nicer default font (+ math font) than Computer Modern for most use cases
    \usepackage{mathpazo}

    % Basic figure setup, for now with no caption control since it's done
    % automatically by Pandoc (which extracts ![](path) syntax from Markdown).
    \usepackage{graphicx}
    % We will generate all images so they have a width \maxwidth. This means
    % that they will get their normal width if they fit onto the page, but
    % are scaled down if they would overflow the margins.
    \makeatletter
    \def\maxwidth{\ifdim\Gin@nat@width>\linewidth\linewidth
    \else\Gin@nat@width\fi}
    \makeatother
    \let\Oldincludegraphics\includegraphics
    % Set max figure width to be 80% of text width, for now hardcoded.
    \renewcommand{\includegraphics}[1]{\Oldincludegraphics[width=.8\maxwidth]{#1}}
    % Ensure that by default, figures have no caption (until we provide a
    % proper Figure object with a Caption API and a way to capture that
    % in the conversion process - todo).
    \usepackage{caption}
    \DeclareCaptionLabelFormat{nolabel}{}
    \captionsetup{labelformat=nolabel}

    \usepackage{adjustbox} % Used to constrain images to a maximum size 
    \usepackage{xcolor} % Allow colors to be defined
    \usepackage{enumerate} % Needed for markdown enumerations to work
    \usepackage{geometry} % Used to adjust the document margins
    \usepackage{amsmath} % Equations
    \usepackage{amssymb} % Equations
    \usepackage{textcomp} % defines textquotesingle
    % Hack from http://tex.stackexchange.com/a/47451/13684:
    \AtBeginDocument{%
        \def\PYZsq{\textquotesingle}% Upright quotes in Pygmentized code
    }
    \usepackage{upquote} % Upright quotes for verbatim code
    \usepackage{eurosym} % defines \euro
    \usepackage[mathletters]{ucs} % Extended unicode (utf-8) support
    \usepackage[utf8x]{inputenc} % Allow utf-8 characters in the tex document
    \usepackage{fancyvrb} % verbatim replacement that allows latex
    \usepackage{grffile} % extends the file name processing of package graphics 
                         % to support a larger range 
    % The hyperref package gives us a pdf with properly built
    % internal navigation ('pdf bookmarks' for the table of contents,
    % internal cross-reference links, web links for URLs, etc.)
    \usepackage{hyperref}
    \usepackage{longtable} % longtable support required by pandoc >1.10
    \usepackage{booktabs}  % table support for pandoc > 1.12.2
    \usepackage[inline]{enumitem} % IRkernel/repr support (it uses the enumerate* environment)
    \usepackage[normalem]{ulem} % ulem is needed to support strikethroughs (\sout)
                                % normalem makes italics be italics, not underlines
    

    
    
    % Colors for the hyperref package
    \definecolor{urlcolor}{rgb}{0,.145,.698}
    \definecolor{linkcolor}{rgb}{.71,0.21,0.01}
    \definecolor{citecolor}{rgb}{.12,.54,.11}

    % ANSI colors
    \definecolor{ansi-black}{HTML}{3E424D}
    \definecolor{ansi-black-intense}{HTML}{282C36}
    \definecolor{ansi-red}{HTML}{E75C58}
    \definecolor{ansi-red-intense}{HTML}{B22B31}
    \definecolor{ansi-green}{HTML}{00A250}
    \definecolor{ansi-green-intense}{HTML}{007427}
    \definecolor{ansi-yellow}{HTML}{DDB62B}
    \definecolor{ansi-yellow-intense}{HTML}{B27D12}
    \definecolor{ansi-blue}{HTML}{208FFB}
    \definecolor{ansi-blue-intense}{HTML}{0065CA}
    \definecolor{ansi-magenta}{HTML}{D160C4}
    \definecolor{ansi-magenta-intense}{HTML}{A03196}
    \definecolor{ansi-cyan}{HTML}{60C6C8}
    \definecolor{ansi-cyan-intense}{HTML}{258F8F}
    \definecolor{ansi-white}{HTML}{C5C1B4}
    \definecolor{ansi-white-intense}{HTML}{A1A6B2}

    % commands and environments needed by pandoc snippets
    % extracted from the output of `pandoc -s`
    \providecommand{\tightlist}{%
      \setlength{\itemsep}{0pt}\setlength{\parskip}{0pt}}
    \DefineVerbatimEnvironment{Highlighting}{Verbatim}{commandchars=\\\{\}}
    % Add ',fontsize=\small' for more characters per line
    \newenvironment{Shaded}{}{}
    \newcommand{\KeywordTok}[1]{\textcolor[rgb]{0.00,0.44,0.13}{\textbf{{#1}}}}
    \newcommand{\DataTypeTok}[1]{\textcolor[rgb]{0.56,0.13,0.00}{{#1}}}
    \newcommand{\DecValTok}[1]{\textcolor[rgb]{0.25,0.63,0.44}{{#1}}}
    \newcommand{\BaseNTok}[1]{\textcolor[rgb]{0.25,0.63,0.44}{{#1}}}
    \newcommand{\FloatTok}[1]{\textcolor[rgb]{0.25,0.63,0.44}{{#1}}}
    \newcommand{\CharTok}[1]{\textcolor[rgb]{0.25,0.44,0.63}{{#1}}}
    \newcommand{\StringTok}[1]{\textcolor[rgb]{0.25,0.44,0.63}{{#1}}}
    \newcommand{\CommentTok}[1]{\textcolor[rgb]{0.38,0.63,0.69}{\textit{{#1}}}}
    \newcommand{\OtherTok}[1]{\textcolor[rgb]{0.00,0.44,0.13}{{#1}}}
    \newcommand{\AlertTok}[1]{\textcolor[rgb]{1.00,0.00,0.00}{\textbf{{#1}}}}
    \newcommand{\FunctionTok}[1]{\textcolor[rgb]{0.02,0.16,0.49}{{#1}}}
    \newcommand{\RegionMarkerTok}[1]{{#1}}
    \newcommand{\ErrorTok}[1]{\textcolor[rgb]{1.00,0.00,0.00}{\textbf{{#1}}}}
    \newcommand{\NormalTok}[1]{{#1}}
    
    % Additional commands for more recent versions of Pandoc
    \newcommand{\ConstantTok}[1]{\textcolor[rgb]{0.53,0.00,0.00}{{#1}}}
    \newcommand{\SpecialCharTok}[1]{\textcolor[rgb]{0.25,0.44,0.63}{{#1}}}
    \newcommand{\VerbatimStringTok}[1]{\textcolor[rgb]{0.25,0.44,0.63}{{#1}}}
    \newcommand{\SpecialStringTok}[1]{\textcolor[rgb]{0.73,0.40,0.53}{{#1}}}
    \newcommand{\ImportTok}[1]{{#1}}
    \newcommand{\DocumentationTok}[1]{\textcolor[rgb]{0.73,0.13,0.13}{\textit{{#1}}}}
    \newcommand{\AnnotationTok}[1]{\textcolor[rgb]{0.38,0.63,0.69}{\textbf{\textit{{#1}}}}}
    \newcommand{\CommentVarTok}[1]{\textcolor[rgb]{0.38,0.63,0.69}{\textbf{\textit{{#1}}}}}
    \newcommand{\VariableTok}[1]{\textcolor[rgb]{0.10,0.09,0.49}{{#1}}}
    \newcommand{\ControlFlowTok}[1]{\textcolor[rgb]{0.00,0.44,0.13}{\textbf{{#1}}}}
    \newcommand{\OperatorTok}[1]{\textcolor[rgb]{0.40,0.40,0.40}{{#1}}}
    \newcommand{\BuiltInTok}[1]{{#1}}
    \newcommand{\ExtensionTok}[1]{{#1}}
    \newcommand{\PreprocessorTok}[1]{\textcolor[rgb]{0.74,0.48,0.00}{{#1}}}
    \newcommand{\AttributeTok}[1]{\textcolor[rgb]{0.49,0.56,0.16}{{#1}}}
    \newcommand{\InformationTok}[1]{\textcolor[rgb]{0.38,0.63,0.69}{\textbf{\textit{{#1}}}}}
    \newcommand{\WarningTok}[1]{\textcolor[rgb]{0.38,0.63,0.69}{\textbf{\textit{{#1}}}}}
    
    
    % Define a nice break command that doesn't care if a line doesn't already
    % exist.
    \def\br{\hspace*{\fill} \\* }
    % Math Jax compatability definitions
    \def\gt{>}
    \def\lt{<}
    % Document parameters
    \title{300????}
    
    
    

    % Pygments definitions
    
\makeatletter
\def\PY@reset{\let\PY@it=\relax \let\PY@bf=\relax%
    \let\PY@ul=\relax \let\PY@tc=\relax%
    \let\PY@bc=\relax \let\PY@ff=\relax}
\def\PY@tok#1{\csname PY@tok@#1\endcsname}
\def\PY@toks#1+{\ifx\relax#1\empty\else%
    \PY@tok{#1}\expandafter\PY@toks\fi}
\def\PY@do#1{\PY@bc{\PY@tc{\PY@ul{%
    \PY@it{\PY@bf{\PY@ff{#1}}}}}}}
\def\PY#1#2{\PY@reset\PY@toks#1+\relax+\PY@do{#2}}

\expandafter\def\csname PY@tok@w\endcsname{\def\PY@tc##1{\textcolor[rgb]{0.73,0.73,0.73}{##1}}}
\expandafter\def\csname PY@tok@c\endcsname{\let\PY@it=\textit\def\PY@tc##1{\textcolor[rgb]{0.25,0.50,0.50}{##1}}}
\expandafter\def\csname PY@tok@cp\endcsname{\def\PY@tc##1{\textcolor[rgb]{0.74,0.48,0.00}{##1}}}
\expandafter\def\csname PY@tok@k\endcsname{\let\PY@bf=\textbf\def\PY@tc##1{\textcolor[rgb]{0.00,0.50,0.00}{##1}}}
\expandafter\def\csname PY@tok@kp\endcsname{\def\PY@tc##1{\textcolor[rgb]{0.00,0.50,0.00}{##1}}}
\expandafter\def\csname PY@tok@kt\endcsname{\def\PY@tc##1{\textcolor[rgb]{0.69,0.00,0.25}{##1}}}
\expandafter\def\csname PY@tok@o\endcsname{\def\PY@tc##1{\textcolor[rgb]{0.40,0.40,0.40}{##1}}}
\expandafter\def\csname PY@tok@ow\endcsname{\let\PY@bf=\textbf\def\PY@tc##1{\textcolor[rgb]{0.67,0.13,1.00}{##1}}}
\expandafter\def\csname PY@tok@nb\endcsname{\def\PY@tc##1{\textcolor[rgb]{0.00,0.50,0.00}{##1}}}
\expandafter\def\csname PY@tok@nf\endcsname{\def\PY@tc##1{\textcolor[rgb]{0.00,0.00,1.00}{##1}}}
\expandafter\def\csname PY@tok@nc\endcsname{\let\PY@bf=\textbf\def\PY@tc##1{\textcolor[rgb]{0.00,0.00,1.00}{##1}}}
\expandafter\def\csname PY@tok@nn\endcsname{\let\PY@bf=\textbf\def\PY@tc##1{\textcolor[rgb]{0.00,0.00,1.00}{##1}}}
\expandafter\def\csname PY@tok@ne\endcsname{\let\PY@bf=\textbf\def\PY@tc##1{\textcolor[rgb]{0.82,0.25,0.23}{##1}}}
\expandafter\def\csname PY@tok@nv\endcsname{\def\PY@tc##1{\textcolor[rgb]{0.10,0.09,0.49}{##1}}}
\expandafter\def\csname PY@tok@no\endcsname{\def\PY@tc##1{\textcolor[rgb]{0.53,0.00,0.00}{##1}}}
\expandafter\def\csname PY@tok@nl\endcsname{\def\PY@tc##1{\textcolor[rgb]{0.63,0.63,0.00}{##1}}}
\expandafter\def\csname PY@tok@ni\endcsname{\let\PY@bf=\textbf\def\PY@tc##1{\textcolor[rgb]{0.60,0.60,0.60}{##1}}}
\expandafter\def\csname PY@tok@na\endcsname{\def\PY@tc##1{\textcolor[rgb]{0.49,0.56,0.16}{##1}}}
\expandafter\def\csname PY@tok@nt\endcsname{\let\PY@bf=\textbf\def\PY@tc##1{\textcolor[rgb]{0.00,0.50,0.00}{##1}}}
\expandafter\def\csname PY@tok@nd\endcsname{\def\PY@tc##1{\textcolor[rgb]{0.67,0.13,1.00}{##1}}}
\expandafter\def\csname PY@tok@s\endcsname{\def\PY@tc##1{\textcolor[rgb]{0.73,0.13,0.13}{##1}}}
\expandafter\def\csname PY@tok@sd\endcsname{\let\PY@it=\textit\def\PY@tc##1{\textcolor[rgb]{0.73,0.13,0.13}{##1}}}
\expandafter\def\csname PY@tok@si\endcsname{\let\PY@bf=\textbf\def\PY@tc##1{\textcolor[rgb]{0.73,0.40,0.53}{##1}}}
\expandafter\def\csname PY@tok@se\endcsname{\let\PY@bf=\textbf\def\PY@tc##1{\textcolor[rgb]{0.73,0.40,0.13}{##1}}}
\expandafter\def\csname PY@tok@sr\endcsname{\def\PY@tc##1{\textcolor[rgb]{0.73,0.40,0.53}{##1}}}
\expandafter\def\csname PY@tok@ss\endcsname{\def\PY@tc##1{\textcolor[rgb]{0.10,0.09,0.49}{##1}}}
\expandafter\def\csname PY@tok@sx\endcsname{\def\PY@tc##1{\textcolor[rgb]{0.00,0.50,0.00}{##1}}}
\expandafter\def\csname PY@tok@m\endcsname{\def\PY@tc##1{\textcolor[rgb]{0.40,0.40,0.40}{##1}}}
\expandafter\def\csname PY@tok@gh\endcsname{\let\PY@bf=\textbf\def\PY@tc##1{\textcolor[rgb]{0.00,0.00,0.50}{##1}}}
\expandafter\def\csname PY@tok@gu\endcsname{\let\PY@bf=\textbf\def\PY@tc##1{\textcolor[rgb]{0.50,0.00,0.50}{##1}}}
\expandafter\def\csname PY@tok@gd\endcsname{\def\PY@tc##1{\textcolor[rgb]{0.63,0.00,0.00}{##1}}}
\expandafter\def\csname PY@tok@gi\endcsname{\def\PY@tc##1{\textcolor[rgb]{0.00,0.63,0.00}{##1}}}
\expandafter\def\csname PY@tok@gr\endcsname{\def\PY@tc##1{\textcolor[rgb]{1.00,0.00,0.00}{##1}}}
\expandafter\def\csname PY@tok@ge\endcsname{\let\PY@it=\textit}
\expandafter\def\csname PY@tok@gs\endcsname{\let\PY@bf=\textbf}
\expandafter\def\csname PY@tok@gp\endcsname{\let\PY@bf=\textbf\def\PY@tc##1{\textcolor[rgb]{0.00,0.00,0.50}{##1}}}
\expandafter\def\csname PY@tok@go\endcsname{\def\PY@tc##1{\textcolor[rgb]{0.53,0.53,0.53}{##1}}}
\expandafter\def\csname PY@tok@gt\endcsname{\def\PY@tc##1{\textcolor[rgb]{0.00,0.27,0.87}{##1}}}
\expandafter\def\csname PY@tok@err\endcsname{\def\PY@bc##1{\setlength{\fboxsep}{0pt}\fcolorbox[rgb]{1.00,0.00,0.00}{1,1,1}{\strut ##1}}}
\expandafter\def\csname PY@tok@kc\endcsname{\let\PY@bf=\textbf\def\PY@tc##1{\textcolor[rgb]{0.00,0.50,0.00}{##1}}}
\expandafter\def\csname PY@tok@kd\endcsname{\let\PY@bf=\textbf\def\PY@tc##1{\textcolor[rgb]{0.00,0.50,0.00}{##1}}}
\expandafter\def\csname PY@tok@kn\endcsname{\let\PY@bf=\textbf\def\PY@tc##1{\textcolor[rgb]{0.00,0.50,0.00}{##1}}}
\expandafter\def\csname PY@tok@kr\endcsname{\let\PY@bf=\textbf\def\PY@tc##1{\textcolor[rgb]{0.00,0.50,0.00}{##1}}}
\expandafter\def\csname PY@tok@bp\endcsname{\def\PY@tc##1{\textcolor[rgb]{0.00,0.50,0.00}{##1}}}
\expandafter\def\csname PY@tok@fm\endcsname{\def\PY@tc##1{\textcolor[rgb]{0.00,0.00,1.00}{##1}}}
\expandafter\def\csname PY@tok@vc\endcsname{\def\PY@tc##1{\textcolor[rgb]{0.10,0.09,0.49}{##1}}}
\expandafter\def\csname PY@tok@vg\endcsname{\def\PY@tc##1{\textcolor[rgb]{0.10,0.09,0.49}{##1}}}
\expandafter\def\csname PY@tok@vi\endcsname{\def\PY@tc##1{\textcolor[rgb]{0.10,0.09,0.49}{##1}}}
\expandafter\def\csname PY@tok@vm\endcsname{\def\PY@tc##1{\textcolor[rgb]{0.10,0.09,0.49}{##1}}}
\expandafter\def\csname PY@tok@sa\endcsname{\def\PY@tc##1{\textcolor[rgb]{0.73,0.13,0.13}{##1}}}
\expandafter\def\csname PY@tok@sb\endcsname{\def\PY@tc##1{\textcolor[rgb]{0.73,0.13,0.13}{##1}}}
\expandafter\def\csname PY@tok@sc\endcsname{\def\PY@tc##1{\textcolor[rgb]{0.73,0.13,0.13}{##1}}}
\expandafter\def\csname PY@tok@dl\endcsname{\def\PY@tc##1{\textcolor[rgb]{0.73,0.13,0.13}{##1}}}
\expandafter\def\csname PY@tok@s2\endcsname{\def\PY@tc##1{\textcolor[rgb]{0.73,0.13,0.13}{##1}}}
\expandafter\def\csname PY@tok@sh\endcsname{\def\PY@tc##1{\textcolor[rgb]{0.73,0.13,0.13}{##1}}}
\expandafter\def\csname PY@tok@s1\endcsname{\def\PY@tc##1{\textcolor[rgb]{0.73,0.13,0.13}{##1}}}
\expandafter\def\csname PY@tok@mb\endcsname{\def\PY@tc##1{\textcolor[rgb]{0.40,0.40,0.40}{##1}}}
\expandafter\def\csname PY@tok@mf\endcsname{\def\PY@tc##1{\textcolor[rgb]{0.40,0.40,0.40}{##1}}}
\expandafter\def\csname PY@tok@mh\endcsname{\def\PY@tc##1{\textcolor[rgb]{0.40,0.40,0.40}{##1}}}
\expandafter\def\csname PY@tok@mi\endcsname{\def\PY@tc##1{\textcolor[rgb]{0.40,0.40,0.40}{##1}}}
\expandafter\def\csname PY@tok@il\endcsname{\def\PY@tc##1{\textcolor[rgb]{0.40,0.40,0.40}{##1}}}
\expandafter\def\csname PY@tok@mo\endcsname{\def\PY@tc##1{\textcolor[rgb]{0.40,0.40,0.40}{##1}}}
\expandafter\def\csname PY@tok@ch\endcsname{\let\PY@it=\textit\def\PY@tc##1{\textcolor[rgb]{0.25,0.50,0.50}{##1}}}
\expandafter\def\csname PY@tok@cm\endcsname{\let\PY@it=\textit\def\PY@tc##1{\textcolor[rgb]{0.25,0.50,0.50}{##1}}}
\expandafter\def\csname PY@tok@cpf\endcsname{\let\PY@it=\textit\def\PY@tc##1{\textcolor[rgb]{0.25,0.50,0.50}{##1}}}
\expandafter\def\csname PY@tok@c1\endcsname{\let\PY@it=\textit\def\PY@tc##1{\textcolor[rgb]{0.25,0.50,0.50}{##1}}}
\expandafter\def\csname PY@tok@cs\endcsname{\let\PY@it=\textit\def\PY@tc##1{\textcolor[rgb]{0.25,0.50,0.50}{##1}}}

\def\PYZbs{\char`\\}
\def\PYZus{\char`\_}
\def\PYZob{\char`\{}
\def\PYZcb{\char`\}}
\def\PYZca{\char`\^}
\def\PYZam{\char`\&}
\def\PYZlt{\char`\<}
\def\PYZgt{\char`\>}
\def\PYZsh{\char`\#}
\def\PYZpc{\char`\%}
\def\PYZdl{\char`\$}
\def\PYZhy{\char`\-}
\def\PYZsq{\char`\'}
\def\PYZdq{\char`\"}
\def\PYZti{\char`\~}
% for compatibility with earlier versions
\def\PYZat{@}
\def\PYZlb{[}
\def\PYZrb{]}
\makeatother


    % Exact colors from NB
    \definecolor{incolor}{rgb}{0.0, 0.0, 0.5}
    \definecolor{outcolor}{rgb}{0.545, 0.0, 0.0}



    
    % Prevent overflowing lines due to hard-to-break entities
    \sloppy 
    % Setup hyperref package
    \hypersetup{
      breaklinks=true,  % so long urls are correctly broken across lines
      colorlinks=true,
      urlcolor=urlcolor,
      linkcolor=linkcolor,
      citecolor=citecolor,
      }
    % Slightly bigger margins than the latex defaults
    
    \geometry{verbose,tmargin=1in,bmargin=1in,lmargin=1in,rmargin=1in}
    
    

    \begin{document}
    
    
    \maketitle
    
    

    
    \subsection{人工知能? 機械学習?
深層学習?}\label{ux4ebaux5de5ux77e5ux80fd-ux6a5fux68b0ux5b66ux7fd2-ux6df1ux5c64ux5b66ux7fd2}

(執筆中)
\includegraphics{https://i2.wp.com/www.jamesserra.com/wp-content/uploads/2017/04/AI-Terms.png?resize=1024\%2C642\&ssl=1}
\href{http://www.jamesserra.com/archive/2017/04/artificial-intelligence-defined/}{James
Serraのブログ}より転載 \#\#\# 人工知能(AI)
AIは、論理、if-thenルール、意思決定ツリー、機械学習(深層学習を含む)を使用して、コンピュータが人間の知能を模倣できるようにするあらゆる技術に適用される最も広い用語です。
\#\#\# 機械学習
機械が経験によってタスクを改善することを可能にする、難解な統計的手法を含むAIのサブセットです。カテゴリには深層学習が含まれます。
\#\#\# 深層学習(ディープラーニング)
多層神経ネットワーク(ディープニューラルネットワーク)を大量のデータにさらすことによって、音声や画像認識などのタスクを実行するソフトウェアの学習を可能にするアルゴリズムで構成された機械学習のサブセット。

    \subsection{機械学習の類型}\label{ux6a5fux68b0ux5b66ux7fd2ux306eux985eux578b}

人工知能研究の一分野。\emph{データからモデルを作る装置。}

モデルとは?
「自然科学におけるモデルは、理論を説明するための簡単な具体的なもの。特に幾何学的な図形を用いた概念や物体。」本質を抽出し単純化したもの。
\#\#\# 教師あり学習
入力データと出力データの対応関係を把握する。ラベル付き学習データに基き、未知データのラベルを予測するモデル。

\begin{itemize}
\tightlist
\item
  顔認識
\item
  音声認識
\end{itemize}

人間の行動から学び、まねる装置を作る。 \#\#\# 教師なし学習
ラベルのないデータをカテゴライズするモデル。どんなパターンがありうるかという情報なしに、データの隠れたパターンまたは特徴を発見すること。

例: * ソーシャルネットワーク上の親しい友人の特定 *
クレジットカード不正使用に対する行動異常の検出 *
分類、クラスタリングもたぶんこれに含まれる。データマイニング。

\subsubsection{次元削減}\label{ux6b21ux5143ux524aux6e1b}

高次元データに内在する低次元構造を検出するモデル。

\subsubsection{回帰}\label{ux56deux5e30}

統計学において、\(Y\) が連続値の時にデータに \(Y = f(X)\)
というモデル(「定量的な関係の構造」)を当てはめる事。
別の言い方では、連続尺度の従属変数(目的変数)\(Y\)
と独立変数(説明変数)\(X\) の間にモデルを当てはめること。

例: * 最小二乗法

    \subsection{機械学習の例}\label{ux6a5fux68b0ux5b66ux7fd2ux306eux4f8b}

\subsubsection{教師なし学習の例}\label{ux6559ux5e2bux306aux3057ux5b66ux7fd2ux306eux4f8b}

それぞれの手書き文字の画像は、8x8=64ピクセルの濃淡でできているので、64次元のベクトルとみなせる。これを、多様体学習と呼ばれる手法を用い、2次元に落としこむ。コンピュータはこのデータが6種類の数字でできていることは知らない。64次元空間で近い点(似た画像)は、2次元空間でも近くなるように、射影される。(実際の計算時間は瞬時)

    \begin{Verbatim}[commandchars=\\\{\}]
{\color{incolor}In [{\color{incolor}2}]:} \PY{o}{\PYZpc{}}\PY{k}{matplotlib} inline
        
        \PY{k+kn}{import} \PY{n+nn}{matplotlib}\PY{n+nn}{.}\PY{n+nn}{pyplot} \PY{k}{as} \PY{n+nn}{plt}
        \PY{k+kn}{from} \PY{n+nn}{sklearn}\PY{n+nn}{.}\PY{n+nn}{datasets} \PY{k}{import} \PY{n}{load\PYZus{}digits}
        
        \PY{c+c1}{\PYZsh{} 画像データの読みこみ}
        \PY{n}{digits} \PY{o}{=} \PY{n}{load\PYZus{}digits}\PY{p}{(}\PY{n}{n\PYZus{}class}\PY{o}{=}\PY{l+m+mi}{6}\PY{p}{)}
        
        \PY{c+c1}{\PYZsh{} 画像データの表示}
        \PY{n}{fig}\PY{p}{,} \PY{n}{ax} \PY{o}{=} \PY{n}{plt}\PY{o}{.}\PY{n}{subplots}\PY{p}{(}\PY{l+m+mi}{8}\PY{p}{,} \PY{l+m+mi}{8}\PY{p}{,} \PY{n}{figsize}\PY{o}{=}\PY{p}{(}\PY{l+m+mi}{6}\PY{p}{,} \PY{l+m+mi}{6}\PY{p}{)}\PY{p}{)}
        \PY{k}{for} \PY{n}{i}\PY{p}{,} \PY{n}{axi} \PY{o+ow}{in} \PY{n+nb}{enumerate}\PY{p}{(}\PY{n}{ax}\PY{o}{.}\PY{n}{flat}\PY{p}{)}\PY{p}{:}
            \PY{n}{axi}\PY{o}{.}\PY{n}{imshow}\PY{p}{(}\PY{n}{digits}\PY{o}{.}\PY{n}{images}\PY{p}{[}\PY{n}{i}\PY{p}{]}\PY{p}{,} \PY{n}{cmap}\PY{o}{=}\PY{l+s+s1}{\PYZsq{}}\PY{l+s+s1}{binary}\PY{l+s+s1}{\PYZsq{}}\PY{p}{)}
            \PY{n}{axi}\PY{o}{.}\PY{n}{set}\PY{p}{(}\PY{n}{xticks}\PY{o}{=}\PY{p}{[}\PY{p}{]}\PY{p}{,} \PY{n}{yticks}\PY{o}{=}\PY{p}{[}\PY{p}{]}\PY{p}{)}
\end{Verbatim}


    \begin{center}
    \adjustimage{max size={0.9\linewidth}{0.9\paperheight}}{output_3_0.png}
    \end{center}
    { \hspace*{\fill} \\}
    
    \begin{Verbatim}[commandchars=\\\{\}]
{\color{incolor}In [{\color{incolor}6}]:} \PY{c+c1}{\PYZsh{} IsoMap (多様体学習の一手法; 次元を削減し、データの構造を明らかにする)}
        \PY{k+kn}{from} \PY{n+nn}{sklearn}\PY{n+nn}{.}\PY{n+nn}{manifold} \PY{k}{import} \PY{n}{Isomap}
        \PY{n}{iso} \PY{o}{=} \PY{n}{Isomap}\PY{p}{(}\PY{n}{n\PYZus{}components}\PY{o}{=}\PY{l+m+mi}{2}\PY{p}{)}
        \PY{n}{projection} \PY{o}{=} \PY{n}{iso}\PY{o}{.}\PY{n}{fit\PYZus{}transform}\PY{p}{(}\PY{n}{digits}\PY{o}{.}\PY{n}{data}\PY{p}{)}
        
        \PY{c+c1}{\PYZsh{} 結果の描画}
        \PY{n}{plt}\PY{o}{.}\PY{n}{scatter}\PY{p}{(}\PY{n}{projection}\PY{p}{[}\PY{p}{:}\PY{p}{,} \PY{l+m+mi}{0}\PY{p}{]}\PY{p}{,} \PY{n}{projection}\PY{p}{[}\PY{p}{:}\PY{p}{,} \PY{l+m+mi}{1}\PY{p}{]}\PY{p}{,} \PY{n}{lw}\PY{o}{=}\PY{l+m+mf}{0.1}\PY{p}{,}
                   \PY{n}{c}\PY{o}{=}\PY{l+s+s1}{\PYZsq{}}\PY{l+s+s1}{black}\PY{l+s+s1}{\PYZsq{}}\PY{p}{)}
\end{Verbatim}


\begin{Verbatim}[commandchars=\\\{\}]
{\color{outcolor}Out[{\color{outcolor}6}]:} <matplotlib.collections.PathCollection at 0x11838f198>
\end{Verbatim}
            
    \begin{center}
    \adjustimage{max size={0.9\linewidth}{0.9\paperheight}}{output_4_1.png}
    \end{center}
    { \hspace*{\fill} \\}
    
    それぞれの画像がどの数字を示しているかという情報もデータに含まれているが、コンピュータはそれを使わずにマッピングを行った。

    その結果に対し、文字の種類ごとに異なる色を彩色してやると、6種類の文字が異なる島に分かれている=識別されていることがわかる。この例では、先験的な知識なしに、プログラムが文字をその形だけで分類できることを示している。

    \begin{Verbatim}[commandchars=\\\{\}]
{\color{incolor}In [{\color{incolor}16}]:} \PY{c+c1}{\PYZsh{} 文字の種類ごとに異なる色を付けて描画}
         \PY{n}{plt}\PY{o}{.}\PY{n}{scatter}\PY{p}{(}\PY{n}{projection}\PY{p}{[}\PY{p}{:}\PY{p}{,} \PY{l+m+mi}{0}\PY{p}{]}\PY{p}{,} \PY{n}{projection}\PY{p}{[}\PY{p}{:}\PY{p}{,} \PY{l+m+mi}{1}\PY{p}{]}\PY{p}{,} \PY{n}{lw}\PY{o}{=}\PY{l+m+mf}{0.1}\PY{p}{,}
                    \PY{n}{c}\PY{o}{=}\PY{n}{digits}\PY{o}{.}\PY{n}{target}\PY{p}{,} \PY{n}{cmap}\PY{o}{=}\PY{n}{plt}\PY{o}{.}\PY{n}{cm}\PY{o}{.}\PY{n}{get\PYZus{}cmap}\PY{p}{(}\PY{l+s+s1}{\PYZsq{}}\PY{l+s+s1}{jet}\PY{l+s+s1}{\PYZsq{}}\PY{p}{,} \PY{l+m+mi}{6}\PY{p}{)}\PY{p}{)}
         \PY{n}{plt}\PY{o}{.}\PY{n}{colorbar}\PY{p}{(}\PY{n}{ticks}\PY{o}{=}\PY{n+nb}{range}\PY{p}{(}\PY{l+m+mi}{6}\PY{p}{)}\PY{p}{,} \PY{n}{label}\PY{o}{=}\PY{l+s+s1}{\PYZsq{}}\PY{l+s+s1}{digit value}\PY{l+s+s1}{\PYZsq{}}\PY{p}{)}
         \PY{n}{plt}\PY{o}{.}\PY{n}{clim}\PY{p}{(}\PY{o}{\PYZhy{}}\PY{l+m+mf}{0.5}\PY{p}{,} \PY{l+m+mf}{5.5}\PY{p}{)}
\end{Verbatim}


    \begin{center}
    \adjustimage{max size={0.9\linewidth}{0.9\paperheight}}{output_7_0.png}
    \end{center}
    { \hspace*{\fill} \\}
    
    ひとたびこのようなテリトリーの地図が得られると、未知の手書き文字を同じようにこの地図にのせることで、それがどの文字に一番近いかを「認識」できるようになる。

    \subsection{参考資料}\label{ux53c2ux8003ux8cc7ux6599}

\begin{itemize}
\tightlist
\item
  Pythonデータサイエンスハンドブック
\item
  パターン認識と機械学習
\end{itemize}

    \subsubsection{教師あり学習の例}\label{ux6559ux5e2bux3042ux308aux5b66ux7fd2ux306eux4f8b}

上の例では、2次元に落としこんだ時に、はっきりとしたクラスターが形成され、異なる文字の間の境界が自ずと明らかになった。しかし、いつもこんなにうまくクラスター化できるとは限らない。

学習の過程で、それぞれの画像がどの数字をあらわしているかをコンピュータに教えてやれば、その情報をもとに、異なる文字の境界線をコンピュータが自動的に定める。このような方法を教師あり学習と呼ぶ。

    \paragraph{ランダムフォレスト法による文字分類}\label{ux30e9ux30f3ux30c0ux30e0ux30d5ux30a9ux30ecux30b9ux30c8ux6cd5ux306bux3088ux308bux6587ux5b57ux5206ux985e}

それぞれの文字の画像と読み方をセットで教え、機械学習アルゴリズムが分類方法を自動的に考える。

    \begin{Verbatim}[commandchars=\\\{\}]
{\color{incolor}In [{\color{incolor}7}]:} \PY{c+c1}{\PYZsh{} taken from Python Data Science handbook}
        
        \PY{o}{\PYZpc{}}\PY{k}{matplotlib} inline
        
        \PY{k+kn}{import} \PY{n+nn}{matplotlib}\PY{n+nn}{.}\PY{n+nn}{pyplot} \PY{k}{as} \PY{n+nn}{plt}
        \PY{k+kn}{from} \PY{n+nn}{sklearn}\PY{n+nn}{.}\PY{n+nn}{datasets} \PY{k}{import} \PY{n}{load\PYZus{}digits}
        \PY{n}{digits} \PY{o}{=} \PY{n}{load\PYZus{}digits}\PY{p}{(}\PY{p}{)}
        
        \PY{c+c1}{\PYZsh{} 画像データの表示}
        \PY{n}{fig}\PY{p}{,} \PY{n}{ax} \PY{o}{=} \PY{n}{plt}\PY{o}{.}\PY{n}{subplots}\PY{p}{(}\PY{l+m+mi}{8}\PY{p}{,} \PY{l+m+mi}{8}\PY{p}{,} \PY{n}{figsize}\PY{o}{=}\PY{p}{(}\PY{l+m+mi}{6}\PY{p}{,} \PY{l+m+mi}{6}\PY{p}{)}\PY{p}{)}
        \PY{n}{fig}\PY{o}{.}\PY{n}{subplots\PYZus{}adjust}\PY{p}{(}\PY{n}{left}\PY{o}{=}\PY{l+m+mi}{0}\PY{p}{,}\PY{n}{right}\PY{o}{=}\PY{l+m+mi}{1}\PY{p}{,} \PY{n}{bottom}\PY{o}{=}\PY{l+m+mi}{0}\PY{p}{,}\PY{n}{top}\PY{o}{=}\PY{l+m+mi}{1}\PY{p}{,}\PY{n}{hspace}\PY{o}{=}\PY{l+m+mf}{0.05}\PY{p}{,}\PY{n}{wspace}\PY{o}{=}\PY{l+m+mf}{0.05}\PY{p}{)}
        \PY{k}{for} \PY{n}{i}\PY{p}{,} \PY{n}{axi} \PY{o+ow}{in} \PY{n+nb}{enumerate}\PY{p}{(}\PY{n}{ax}\PY{o}{.}\PY{n}{flat}\PY{p}{)}\PY{p}{:}
            \PY{n}{axi}\PY{o}{.}\PY{n}{imshow}\PY{p}{(}\PY{n}{digits}\PY{o}{.}\PY{n}{images}\PY{p}{[}\PY{n}{i}\PY{p}{]}\PY{p}{,} \PY{n}{cmap}\PY{o}{=}\PY{l+s+s1}{\PYZsq{}}\PY{l+s+s1}{binary}\PY{l+s+s1}{\PYZsq{}}\PY{p}{)}
            \PY{n}{axi}\PY{o}{.}\PY{n}{text}\PY{p}{(}\PY{l+m+mi}{0}\PY{p}{,}\PY{l+m+mi}{7}\PY{p}{,}\PY{n+nb}{str}\PY{p}{(}\PY{n}{digits}\PY{o}{.}\PY{n}{target}\PY{p}{[}\PY{n}{i}\PY{p}{]}\PY{p}{)}\PY{p}{)}
            \PY{n}{axi}\PY{o}{.}\PY{n}{set}\PY{p}{(}\PY{n}{xticks}\PY{o}{=}\PY{p}{[}\PY{p}{]}\PY{p}{,} \PY{n}{yticks}\PY{o}{=}\PY{p}{[}\PY{p}{]}\PY{p}{)}
            
        \PY{c+c1}{\PYZsh{} 学習用データとテストデータの分割}
        \PY{k+kn}{from} \PY{n+nn}{sklearn}\PY{n+nn}{.}\PY{n+nn}{model\PYZus{}selection} \PY{k}{import} \PY{n}{train\PYZus{}test\PYZus{}split}
        \PY{k+kn}{from} \PY{n+nn}{sklearn}\PY{n+nn}{.}\PY{n+nn}{ensemble}        \PY{k}{import} \PY{n}{RandomForestClassifier}
        
        \PY{n}{Xtrain}\PY{p}{,} \PY{n}{Xtest}\PY{p}{,} \PY{n}{ytrain}\PY{p}{,} \PY{n}{ytest} \PY{o}{=} \PY{n}{train\PYZus{}test\PYZus{}split}\PY{p}{(}\PY{n}{digits}\PY{o}{.}\PY{n}{data}\PY{p}{,} \PY{n}{digits}\PY{o}{.}\PY{n}{target}\PY{p}{,} \PY{n}{random\PYZus{}state}\PY{o}{=}\PY{l+m+mi}{0}\PY{p}{)}
        \PY{n}{model} \PY{o}{=} \PY{n}{RandomForestClassifier}\PY{p}{(}\PY{n}{n\PYZus{}estimators}\PY{o}{=}\PY{l+m+mi}{1000}\PY{p}{)}
        \PY{n}{model}\PY{o}{.}\PY{n}{fit}\PY{p}{(}\PY{n}{Xtrain}\PY{p}{,} \PY{n}{ytrain}\PY{p}{)}
\end{Verbatim}


\begin{Verbatim}[commandchars=\\\{\}]
{\color{outcolor}Out[{\color{outcolor}7}]:} RandomForestClassifier(bootstrap=True, class\_weight=None, criterion='gini',
                    max\_depth=None, max\_features='auto', max\_leaf\_nodes=None,
                    min\_impurity\_decrease=0.0, min\_impurity\_split=None,
                    min\_samples\_leaf=1, min\_samples\_split=2,
                    min\_weight\_fraction\_leaf=0.0, n\_estimators=1000, n\_jobs=1,
                    oob\_score=False, random\_state=None, verbose=0,
                    warm\_start=False)
\end{Verbatim}
            
    \begin{center}
    \adjustimage{max size={0.9\linewidth}{0.9\paperheight}}{output_12_1.png}
    \end{center}
    { \hspace*{\fill} \\}
    
    それを使って、未知データの推定を行う。

    \begin{Verbatim}[commandchars=\\\{\}]
{\color{incolor}In [{\color{incolor}28}]:} \PY{n}{ypred} \PY{o}{=} \PY{n}{model}\PY{o}{.}\PY{n}{predict}\PY{p}{(}\PY{n}{Xtest}\PY{p}{)}
         
         \PY{c+c1}{\PYZsh{} print(digits.images)}
         \PY{c+c1}{\PYZsh{} print(Xtest)}
         \PY{c+c1}{\PYZsh{} 画像データの表示}
         \PY{n}{fig}\PY{p}{,} \PY{n}{ax} \PY{o}{=} \PY{n}{plt}\PY{o}{.}\PY{n}{subplots}\PY{p}{(}\PY{l+m+mi}{12}\PY{p}{,} \PY{l+m+mi}{12}\PY{p}{,} \PY{n}{figsize}\PY{o}{=}\PY{p}{(}\PY{l+m+mi}{6}\PY{p}{,} \PY{l+m+mi}{6}\PY{p}{)}\PY{p}{)}
         \PY{n}{fig}\PY{o}{.}\PY{n}{subplots\PYZus{}adjust}\PY{p}{(}\PY{n}{left}\PY{o}{=}\PY{l+m+mi}{0}\PY{p}{,}\PY{n}{right}\PY{o}{=}\PY{l+m+mi}{1}\PY{p}{,} \PY{n}{bottom}\PY{o}{=}\PY{l+m+mi}{0}\PY{p}{,}\PY{n}{top}\PY{o}{=}\PY{l+m+mi}{1}\PY{p}{,}\PY{n}{hspace}\PY{o}{=}\PY{l+m+mf}{0.05}\PY{p}{,}\PY{n}{wspace}\PY{o}{=}\PY{l+m+mf}{0.05}\PY{p}{)}
         \PY{k}{for} \PY{n}{i}\PY{p}{,} \PY{n}{axi} \PY{o+ow}{in} \PY{n+nb}{enumerate}\PY{p}{(}\PY{n}{ax}\PY{o}{.}\PY{n}{flat}\PY{p}{)}\PY{p}{:}
             \PY{n}{axi}\PY{o}{.}\PY{n}{set}\PY{p}{(}\PY{n}{xticks}\PY{o}{=}\PY{p}{[}\PY{p}{]}\PY{p}{,} \PY{n}{yticks}\PY{o}{=}\PY{p}{[}\PY{p}{]}\PY{p}{)}
         
             \PY{k}{if} \PY{n}{ypred}\PY{p}{[}\PY{n}{i}\PY{p}{]} \PY{o}{!=} \PY{n}{ytest}\PY{p}{[}\PY{n}{i}\PY{p}{]}\PY{p}{:}
                 \PY{n}{axi}\PY{o}{.}\PY{n}{text}\PY{p}{(}\PY{l+m+mi}{0}\PY{p}{,}\PY{l+m+mi}{7}\PY{p}{,}\PY{n+nb}{str}\PY{p}{(}\PY{n}{ypred}\PY{p}{[}\PY{n}{i}\PY{p}{]}\PY{p}{)}\PY{p}{,} \PY{n}{color}\PY{o}{=}\PY{l+s+s1}{\PYZsq{}}\PY{l+s+s1}{red}\PY{l+s+s1}{\PYZsq{}}\PY{p}{)}
                 \PY{n}{axi}\PY{o}{.}\PY{n}{imshow}\PY{p}{(}\PY{p}{(}\PY{n}{Xtest}\PY{p}{[}\PY{n}{i}\PY{p}{]}\PY{p}{)}\PY{o}{.}\PY{n}{reshape}\PY{p}{(}\PY{l+m+mi}{8}\PY{p}{,}\PY{l+m+mi}{8}\PY{p}{)}\PY{p}{,} \PY{n}{cmap}\PY{o}{=}\PY{l+s+s1}{\PYZsq{}}\PY{l+s+s1}{Reds}\PY{l+s+s1}{\PYZsq{}}\PY{p}{)}
             \PY{k}{else}\PY{p}{:}
                 \PY{n}{axi}\PY{o}{.}\PY{n}{imshow}\PY{p}{(}\PY{n}{Xtest}\PY{p}{[}\PY{n}{i}\PY{p}{]}\PY{o}{.}\PY{n}{reshape}\PY{p}{(}\PY{l+m+mi}{8}\PY{p}{,}\PY{l+m+mi}{8}\PY{p}{)}\PY{p}{,} \PY{n}{cmap}\PY{o}{=}\PY{l+s+s1}{\PYZsq{}}\PY{l+s+s1}{binary}\PY{l+s+s1}{\PYZsq{}}\PY{p}{)}
         
         
         
         \PY{k+kn}{from} \PY{n+nn}{sklearn} \PY{k}{import} \PY{n}{metrics}
         \PY{n+nb}{print}\PY{p}{(}\PY{n}{metrics}\PY{o}{.}\PY{n}{classification\PYZus{}report}\PY{p}{(}\PY{n}{ypred}\PY{p}{,} \PY{n}{ytest}\PY{p}{)}\PY{p}{)}
\end{Verbatim}


    \begin{Verbatim}[commandchars=\\\{\}]
             precision    recall  f1-score   support

          0       1.00      0.97      0.99        38
          1       0.98      0.98      0.98        43
          2       0.95      1.00      0.98        42
          3       0.98      0.98      0.98        45
          4       0.97      1.00      0.99        37
          5       0.98      0.96      0.97        49
          6       1.00      1.00      1.00        52
          7       1.00      0.98      0.99        49
          8       0.98      0.98      0.98        48
          9       0.98      0.98      0.98        47

avg / total       0.98      0.98      0.98       450


    \end{Verbatim}

    \begin{center}
    \adjustimage{max size={0.9\linewidth}{0.9\paperheight}}{output_14_1.png}
    \end{center}
    { \hspace*{\fill} \\}
    
    \subsection{ニューラルネットワーク入門}\label{ux30cbux30e5ux30fcux30e9ux30ebux30cdux30c3ux30c8ux30efux30fcux30afux5165ux9580}

人工ニューラルネットワーク(ANN,
人工神経回路網)は、近年になって急激に脚光を浴びはじめた。

\begin{enumerate}
\def\labelenumi{\arabic{enumi}.}
\item
  \textbf{これまでの手法の限界}
  教える(=認識する)内容が複雑になるにつれ、人によるパラメータチューニングに限界が見えてきた。
\item
  \textbf{NNにおけるブレークスルー}
  NN自体は1940年代から研究が始まっていたが、性能が伸び悩み、実用には向かないと考えられていた。今世紀に入りそれを打破する新しいアイディアにより、人の手助けなしに、自律的に極めて高度な学習ができるようになった。
\item
  \textbf{データ量の増大}
  デジタル写真や音声データが一般化し、膨大な教師データを準備できるようになった。
\item
  \textbf{計算機性能の向上} 膨大な量のトレーニングができるようになった。
\end{enumerate}

\subsubsection{ANNとは何か}\label{annux3068ux306fux4f55ux304b}

神経細胞(ニューロン)の信号伝達をモデル化し、それを大量に組みあわせてネットワーク化して脳の信号処理を模倣する試み。脳機能を忠実に再現しているわけではない。

\subsubsection{何ができるのか}\label{ux4f55ux304cux3067ux304dux308bux306eux304b}

画像や音声などの、大きなデータを入力し、分類、エンコーディング、認識、タグ付け、ノイズ除去などのさまざまなタスクを行う。最新のディープニューラルネットワークを用いると、画像認識、文字認識等で人間を越える能力を発揮する。

\subsubsection{ディープラーニングとは何か}\label{ux30c7ux30a3ux30fcux30d7ux30e9ux30fcux30cbux30f3ux30b0ux3068ux306fux4f55ux304b}

ディープニューラルネットワークを用いた機械学習。

\subsubsection{ディープニューラルネットワークとは何か}\label{ux30c7ux30a3ux30fcux30d7ux30cbux30e5ux30fcux30e9ux30ebux30cdux30c3ux30c8ux30efux30fcux30afux3068ux306fux4f55ux304b}

ネットワークの階層性が深いニューラルネットワークのこと。2016年現在、「非常に深い」ニューラルネットワークとは
\emph{10層}
程度を指していたが、2018年にはすでに何十層もあるのがあたりまえになりつつある。基本的な構造はどのニューラルネットワークも同じだが、ディープニューラルネットワークを設計する場合には、各階層に役割を与える場合が多い。

\subsubsection{もてはやされる理由}\label{ux3082ux3066ux306fux3084ux3055ux308cux308bux7406ux7531}

\begin{enumerate}
\def\labelenumi{\arabic{enumi}.}
\item
  人力では達成不可能な複雑なタスクを実現できるようになった。
\item
  学習過程での計算量は莫大だが、一旦学習したあとは、かけ算と足し算の組みあわせによる、わずかな計算しか必要としない
  = スマホ程度の計算能力で十分利用できる。
\end{enumerate}

\subsubsection{我々科学者の立ち位置}\label{ux6211ux3005ux79d1ux5b66ux8005ux306eux7acbux3061ux4f4dux7f6e}

\begin{enumerate}
\def\labelenumi{\arabic{enumi}.}
\tightlist
\item
  DNNの開発に参入するのは難しい。計算機シミュレーションのために、半導体設計から始めるようなもの。よほどの勝算がなければやめておいたほうがいい。
\item
  DNNで何ができ、何ができないかを理解した上で、DNNに適合するような問題設定を考える。既存のフレームワーク(TensorFlowなど)を使って自力で解くか、外部専門家に適切な要望を出して解決してもらう。
\end{enumerate}

\subsubsection{フレームワーク}\label{ux30d5ux30ecux30fcux30e0ux30efux30fcux30af}

フレームワークとは、ニューラルネットワークを計算機に実装する一連の作業に必要なツール一式のこと。フレームワークが違うと、重みやバイアスといった変数をどんな形式で指定するか、バックプロパゲーションをどのように実施するか、などといった作業手順が違ってくる。

適切なフレームワークを選べば、計算機の性能を最大限にひきだせる。複数のGPUを使って計算を加速したり、外部のCloudに任せてしまうことすら可能。(例えばAmazonのクラウドAWSはMXNetやTensorFlowに対応しているらしい)

(July 2018)

\begin{longtable}[]{@{}cclll@{}}
\toprule
FW name & 公開年 & 開発主体 & 言語 & github\tabularnewline
\midrule
\endhead
Theano & (2010) & MontrealU & Python &
\href{https://github.com/Theano/Theano}{Theano/Theano}\tabularnewline
Caffe & 2013 & UCB & C++, Python &
\href{https://github.com/BVLC/caffe}{BVLC/caffe}\tabularnewline
Chainer & 2015 & PFN & Python &
\href{https://github.com/chainer/chainer}{chainer/chainer}\tabularnewline
CNTK & 2015 & Microsoft & Python &
\href{https://github.com/Microsoft/CNTK}{Microsoft/CNTK}\tabularnewline
TensorFlow & 2015 & Google & C++, Python &
\href{https://github.com/tensorflow/tensorflow}{tensorflow/tensorflow}\tabularnewline
neon & 2015 & Intel & Python &
\href{https://github.com/NervanaSystems/neon}{NervanaSystems/neon}\tabularnewline
PyTorch & 2016 & NYU, Facebook & Python &
\href{https://github.com/pytorch/pytorch}{pytorch/pytorch}\tabularnewline
NNabla & 2017 & Sony & C++, Python &
\href{https://github.com/sony/nnabla}{sony/nnabla}\tabularnewline
Caffe2 & 2017 & Facebook & C++, Python &
\href{https://github.com/caffe2/caffe2}{caffe2/caffe2}\tabularnewline
MXNet & 2017 & WU, CMU & Python, R, Julia, Go &
\href{https://github.com/apache/incubator-mxnet}{apache/incubator-mxnet}\tabularnewline
\bottomrule
\end{longtable}

\begin{itemize}
\tightlist
\item
  Pythonが書ければ、どれも使える。
\item
  どのプロジェクトも、\href{https://github.com}{github}でソースプログラムを公開している。
\item
  https://qiita.com/bonotake/items/cbd44abbcbe333f264d8 などを参考に整理
\end{itemize}

    \subsection{TensorFlowによる実装例}\label{tensorflowux306bux3088ux308bux5b9fux88c5ux4f8b}

https://www.tensorflow.org/versions/r1.0/get\_started/mnist/beginners
をなぞってみる。

    \begin{Verbatim}[commandchars=\\\{\}]
{\color{incolor}In [{\color{incolor}1}]:} \PY{k+kn}{from} \PY{n+nn}{tensorflow}\PY{n+nn}{.}\PY{n+nn}{examples}\PY{n+nn}{.}\PY{n+nn}{tutorials}\PY{n+nn}{.}\PY{n+nn}{mnist} \PY{k}{import} \PY{n}{input\PYZus{}data}
        \PY{n}{mnist} \PY{o}{=} \PY{n}{input\PYZus{}data}\PY{o}{.}\PY{n}{read\PYZus{}data\PYZus{}sets}\PY{p}{(}\PY{l+s+s2}{\PYZdq{}}\PY{l+s+s2}{MNIST\PYZus{}data/}\PY{l+s+s2}{\PYZdq{}}\PY{p}{,} \PY{n}{one\PYZus{}hot}\PY{o}{=}\PY{k+kc}{True}\PY{p}{)}
\end{Verbatim}


    \begin{Verbatim}[commandchars=\\\{\}]
/anaconda2/lib/python2.7/site-packages/h5py/\_\_init\_\_.py:36: FutureWarning: Conversion of the second argument of issubdtype from `float` to `np.floating` is deprecated. In future, it will be treated as `np.float64 == np.dtype(float).type`.
  from .\_conv import register\_converters as \_register\_converters

    \end{Verbatim}

    \begin{Verbatim}[commandchars=\\\{\}]
WARNING:tensorflow:From <ipython-input-1-69c65344baec>:2: read\_data\_sets (from tensorflow.contrib.learn.python.learn.datasets.mnist) is deprecated and will be removed in a future version.
Instructions for updating:
Please use alternatives such as official/mnist/dataset.py from tensorflow/models.
WARNING:tensorflow:From /anaconda2/lib/python2.7/site-packages/tensorflow/contrib/learn/python/learn/datasets/mnist.py:260: maybe\_download (from tensorflow.contrib.learn.python.learn.datasets.base) is deprecated and will be removed in a future version.
Instructions for updating:
Please write your own downloading logic.
WARNING:tensorflow:From /anaconda2/lib/python2.7/site-packages/tensorflow/contrib/learn/python/learn/datasets/mnist.py:262: extract\_images (from tensorflow.contrib.learn.python.learn.datasets.mnist) is deprecated and will be removed in a future version.
Instructions for updating:
Please use tf.data to implement this functionality.
Extracting MNIST\_data/train-images-idx3-ubyte.gz
WARNING:tensorflow:From /anaconda2/lib/python2.7/site-packages/tensorflow/contrib/learn/python/learn/datasets/mnist.py:267: extract\_labels (from tensorflow.contrib.learn.python.learn.datasets.mnist) is deprecated and will be removed in a future version.
Instructions for updating:
Please use tf.data to implement this functionality.
Extracting MNIST\_data/train-labels-idx1-ubyte.gz
WARNING:tensorflow:From /anaconda2/lib/python2.7/site-packages/tensorflow/contrib/learn/python/learn/datasets/mnist.py:110: dense\_to\_one\_hot (from tensorflow.contrib.learn.python.learn.datasets.mnist) is deprecated and will be removed in a future version.
Instructions for updating:
Please use tf.one\_hot on tensors.
Extracting MNIST\_data/t10k-images-idx3-ubyte.gz
Extracting MNIST\_data/t10k-labels-idx1-ubyte.gz
WARNING:tensorflow:From /anaconda2/lib/python2.7/site-packages/tensorflow/contrib/learn/python/learn/datasets/mnist.py:290: \_\_init\_\_ (from tensorflow.contrib.learn.python.learn.datasets.mnist) is deprecated and will be removed in a future version.
Instructions for updating:
Please use alternatives such as official/mnist/dataset.py from tensorflow/models.

    \end{Verbatim}

    \begin{Verbatim}[commandchars=\\\{\}]
{\color{incolor}In [{\color{incolor}31}]:} \PY{k+kn}{import} \PY{n+nn}{tensorflow} \PY{k}{as} \PY{n+nn}{tf}
\end{Verbatim}


    \begin{Verbatim}[commandchars=\\\{\}]
{\color{incolor}In [{\color{incolor}76}]:} \PY{c+c1}{\PYZsh{} 1段だけのNNを準備する (遅延評価)}
         \PY{n}{x} \PY{o}{=} \PY{n}{tf}\PY{o}{.}\PY{n}{placeholder}\PY{p}{(}\PY{n}{tf}\PY{o}{.}\PY{n}{float32}\PY{p}{,} \PY{p}{[}\PY{k+kc}{None}\PY{p}{,} \PY{l+m+mi}{784}\PY{p}{]}\PY{p}{)}
         
         \PY{n}{W} \PY{o}{=} \PY{n}{tf}\PY{o}{.}\PY{n}{Variable}\PY{p}{(}\PY{n}{tf}\PY{o}{.}\PY{n}{zeros}\PY{p}{(}\PY{p}{[}\PY{l+m+mi}{784}\PY{p}{,} \PY{l+m+mi}{10}\PY{p}{]}\PY{p}{)}\PY{p}{)}
         \PY{n}{b} \PY{o}{=} \PY{n}{tf}\PY{o}{.}\PY{n}{Variable}\PY{p}{(}\PY{n}{tf}\PY{o}{.}\PY{n}{zeros}\PY{p}{(}\PY{p}{[}\PY{l+m+mi}{10}\PY{p}{]}\PY{p}{)}\PY{p}{)}
         
         \PY{n}{y} \PY{o}{=} \PY{n}{tf}\PY{o}{.}\PY{n}{nn}\PY{o}{.}\PY{n}{softmax}\PY{p}{(}\PY{n}{tf}\PY{o}{.}\PY{n}{matmul}\PY{p}{(}\PY{n}{x}\PY{p}{,} \PY{n}{W}\PY{p}{)} \PY{o}{+} \PY{n}{b}\PY{p}{)}
\end{Verbatim}


\begin{Verbatim}[commandchars=\\\{\}]
{\color{outcolor}Out[{\color{outcolor}76}]:} <tf.Tensor 'Softmax\_2:0' shape=(?, 10) dtype=float32>
\end{Verbatim}
            
    \begin{Verbatim}[commandchars=\\\{\}]
{\color{incolor}In [{\color{incolor}35}]:} \PY{n}{y\PYZus{}} \PY{o}{=} \PY{n}{tf}\PY{o}{.}\PY{n}{placeholder}\PY{p}{(}\PY{n}{tf}\PY{o}{.}\PY{n}{float32}\PY{p}{,} \PY{p}{[}\PY{k+kc}{None}\PY{p}{,} \PY{l+m+mi}{10}\PY{p}{]}\PY{p}{)}
\end{Verbatim}


    \begin{Verbatim}[commandchars=\\\{\}]
{\color{incolor}In [{\color{incolor}77}]:} \PY{c+c1}{\PYZsh{} 評価関数の定義 (遅延評価)}
         \PY{n}{cross\PYZus{}entropy} \PY{o}{=} \PY{n}{tf}\PY{o}{.}\PY{n}{reduce\PYZus{}mean}\PY{p}{(}\PY{o}{\PYZhy{}}\PY{n}{tf}\PY{o}{.}\PY{n}{reduce\PYZus{}sum}\PY{p}{(}\PY{n}{y\PYZus{}} \PY{o}{*} \PY{n}{tf}\PY{o}{.}\PY{n}{log}\PY{p}{(}\PY{n}{y}\PY{p}{)}\PY{p}{,}
                                                       \PY{n}{reduction\PYZus{}indices}\PY{o}{=}\PY{p}{[}\PY{l+m+mi}{1}\PY{p}{]}\PY{p}{)}\PY{p}{)}
\end{Verbatim}


\begin{Verbatim}[commandchars=\\\{\}]
{\color{outcolor}Out[{\color{outcolor}77}]:} <tf.Tensor 'Mean\_9:0' shape=() dtype=float32>
\end{Verbatim}
            
    \begin{Verbatim}[commandchars=\\\{\}]
{\color{incolor}In [{\color{incolor}79}]:} \PY{c+c1}{\PYZsh{} 訓練方法の定義 (訓練データによる評価→バックプロパゲーション。遅延評価)}
         \PY{n}{train\PYZus{}step} \PY{o}{=} \PY{n}{tf}\PY{o}{.}\PY{n}{train}\PY{o}{.}\PY{n}{GradientDescentOptimizer}\PY{p}{(}\PY{l+m+mf}{0.5}\PY{p}{)}\PY{o}{.}\PY{n}{minimize}\PY{p}{(}\PY{n}{cross\PYZus{}entropy}\PY{p}{)}
\end{Verbatim}


\begin{Verbatim}[commandchars=\\\{\}]
{\color{outcolor}Out[{\color{outcolor}79}]:} <tf.Operation 'GradientDescent\_3' type=NoOp>
\end{Verbatim}
            
    \begin{Verbatim}[commandchars=\\\{\}]
{\color{incolor}In [{\color{incolor}86}]:} \PY{c+c1}{\PYZsh{} 訓練セッションを作る。}
         \PY{n}{sess} \PY{o}{=} \PY{n}{tf}\PY{o}{.}\PY{n}{InteractiveSession}\PY{p}{(}\PY{p}{)}
\end{Verbatim}


    \begin{Verbatim}[commandchars=\\\{\}]
/anaconda2/envs/base3/lib/python3.6/site-packages/tensorflow/python/client/session.py:1711: UserWarning: An interactive session is already active. This can cause out-of-memory errors in some cases. You must explicitly call `InteractiveSession.close()` to release resources held by the other session(s).
  warnings.warn('An interactive session is already active. This can '

    \end{Verbatim}

    \begin{Verbatim}[commandchars=\\\{\}]
{\color{incolor}In [{\color{incolor}92}]:} \PY{c+c1}{\PYZsh{} 変数を初期化する}
         \PY{n}{tf}\PY{o}{.}\PY{n}{global\PYZus{}variables\PYZus{}initializer}\PY{p}{(}\PY{p}{)}\PY{o}{.}\PY{n}{run}\PY{p}{(}\PY{p}{)}
\end{Verbatim}


    \begin{Verbatim}[commandchars=\\\{\}]
{\color{incolor}In [{\color{incolor}94}]:} \PY{c+c1}{\PYZsh{} 訓練セッションの実行: 訓練データで1000回訓練する。}
         \PY{k}{for} \PY{n}{\PYZus{}} \PY{o+ow}{in} \PY{n+nb}{range}\PY{p}{(}\PY{l+m+mi}{100}\PY{p}{)}\PY{p}{:}
             \PY{c+c1}{\PYZsh{} 100組のランダムなデータを選ぶ}
             \PY{n}{batch\PYZus{}xs}\PY{p}{,} \PY{n}{batch\PYZus{}ys} \PY{o}{=} \PY{n}{mnist}\PY{o}{.}\PY{n}{train}\PY{o}{.}\PY{n}{next\PYZus{}batch}\PY{p}{(}\PY{l+m+mi}{1000}\PY{p}{)}
             \PY{n}{sess}\PY{o}{.}\PY{n}{run}\PY{p}{(}\PY{n}{train\PYZus{}step}\PY{p}{,} \PY{n}{feed\PYZus{}dict}\PY{o}{=}\PY{p}{\PYZob{}}\PY{n}{x}\PY{p}{:} \PY{n}{batch\PYZus{}xs}\PY{p}{,} \PY{n}{y\PYZus{}}\PY{p}{:} \PY{n}{batch\PYZus{}ys}\PY{p}{\PYZcb{}}\PY{p}{)}
\end{Verbatim}


    \begin{Verbatim}[commandchars=\\\{\}]
{\color{incolor}In [{\color{incolor}95}]:} \PY{c+c1}{\PYZsh{} 性能評価関数の定義 (遅延評価):}
         \PY{c+c1}{\PYZsh{} y\PYZus{} (正解)とy (推定)の、それぞれの最高確率のものが一致するかどうか}
         \PY{n}{correct\PYZus{}prediction} \PY{o}{=} \PY{n}{tf}\PY{o}{.}\PY{n}{equal}\PY{p}{(}\PY{n}{tf}\PY{o}{.}\PY{n}{argmax}\PY{p}{(}\PY{n}{y}\PY{p}{,}\PY{l+m+mi}{1}\PY{p}{)}\PY{p}{,} \PY{n}{tf}\PY{o}{.}\PY{n}{argmax}\PY{p}{(}\PY{n}{y\PYZus{}}\PY{p}{,}\PY{l+m+mi}{1}\PY{p}{)}\PY{p}{)}
\end{Verbatim}


    \begin{Verbatim}[commandchars=\\\{\}]
{\color{incolor}In [{\color{incolor}96}]:} \PY{c+c1}{\PYZsh{} 精度関数の定義 (遅延評価)}
         \PY{n}{accuracy} \PY{o}{=} \PY{n}{tf}\PY{o}{.}\PY{n}{reduce\PYZus{}mean}\PY{p}{(}\PY{n}{tf}\PY{o}{.}\PY{n}{cast}\PY{p}{(}\PY{n}{correct\PYZus{}prediction}\PY{p}{,} \PY{n}{tf}\PY{o}{.}\PY{n}{float32}\PY{p}{)}\PY{p}{)}
\end{Verbatim}


    \begin{Verbatim}[commandchars=\\\{\}]
{\color{incolor}In [{\color{incolor}97}]:} \PY{c+c1}{\PYZsh{} 検証セッションの実行}
         \PY{n+nb}{print}\PY{p}{(}\PY{n}{sess}\PY{o}{.}\PY{n}{run}\PY{p}{(}\PY{n}{accuracy}\PY{p}{,} 
                        \PY{n}{feed\PYZus{}dict}\PY{o}{=}\PY{p}{\PYZob{}}\PY{n}{x}\PY{p}{:} \PY{n}{mnist}\PY{o}{.}\PY{n}{test}\PY{o}{.}\PY{n}{images}\PY{p}{,}
                                   \PY{n}{y\PYZus{}}\PY{p}{:} \PY{n}{mnist}\PY{o}{.}\PY{n}{test}\PY{o}{.}\PY{n}{labels}\PY{p}{\PYZcb{}}\PY{p}{)}\PY{p}{)}
\end{Verbatim}


    \begin{Verbatim}[commandchars=\\\{\}]
0.9213

    \end{Verbatim}

    テストデータセットでの正答率は92\%。悪い。

    \begin{Verbatim}[commandchars=\\\{\}]
{\color{incolor}In [{\color{incolor}98}]:} \PY{n}{sess}\PY{o}{.}\PY{n}{close}\PY{p}{(}\PY{p}{)}
\end{Verbatim}


    \subsubsection{性能改善}\label{ux6027ux80fdux6539ux5584}

どうすれば良いか? * 階層を増やす。 *
階層を増やすと過学習に陥る可能性がある。 * トレーニングを増やす。 *
dropout(いくつかのリンクをランダムに外し、外乱に強いrobustなニューラルネットワークに育てる)

    \begin{Verbatim}[commandchars=\\\{\}]
{\color{incolor}In [{\color{incolor}1}]:} \PY{k+kn}{from} \PY{n+nn}{tensorflow}\PY{n+nn}{.}\PY{n+nn}{examples}\PY{n+nn}{.}\PY{n+nn}{tutorials}\PY{n+nn}{.}\PY{n+nn}{mnist} \PY{k}{import} \PY{n}{input\PYZus{}data}
        \PY{n}{mnist} \PY{o}{=} \PY{n}{input\PYZus{}data}\PY{o}{.}\PY{n}{read\PYZus{}data\PYZus{}sets}\PY{p}{(}\PY{l+s+s1}{\PYZsq{}}\PY{l+s+s1}{MNIST\PYZus{}data}\PY{l+s+s1}{\PYZsq{}}\PY{p}{,} \PY{n}{one\PYZus{}hot}\PY{o}{=}\PY{k+kc}{True}\PY{p}{)}
\end{Verbatim}


    \begin{Verbatim}[commandchars=\\\{\}]
/Users/matto/anaconda3/lib/python3.6/site-packages/h5py/\_\_init\_\_.py:36: FutureWarning: Conversion of the second argument of issubdtype from `float` to `np.floating` is deprecated. In future, it will be treated as `np.float64 == np.dtype(float).type`.
  from .\_conv import register\_converters as \_register\_converters

    \end{Verbatim}

    \begin{Verbatim}[commandchars=\\\{\}]
WARNING:tensorflow:From <ipython-input-1-93d8da72a918>:2: read\_data\_sets (from tensorflow.contrib.learn.python.learn.datasets.mnist) is deprecated and will be removed in a future version.
Instructions for updating:
Please use alternatives such as official/mnist/dataset.py from tensorflow/models.
WARNING:tensorflow:From /Users/matto/anaconda3/lib/python3.6/site-packages/tensorflow/contrib/learn/python/learn/datasets/mnist.py:260: maybe\_download (from tensorflow.contrib.learn.python.learn.datasets.base) is deprecated and will be removed in a future version.
Instructions for updating:
Please write your own downloading logic.
WARNING:tensorflow:From /Users/matto/anaconda3/lib/python3.6/site-packages/tensorflow/contrib/learn/python/learn/datasets/mnist.py:262: extract\_images (from tensorflow.contrib.learn.python.learn.datasets.mnist) is deprecated and will be removed in a future version.
Instructions for updating:
Please use tf.data to implement this functionality.
Extracting MNIST\_data/train-images-idx3-ubyte.gz
WARNING:tensorflow:From /Users/matto/anaconda3/lib/python3.6/site-packages/tensorflow/contrib/learn/python/learn/datasets/mnist.py:267: extract\_labels (from tensorflow.contrib.learn.python.learn.datasets.mnist) is deprecated and will be removed in a future version.
Instructions for updating:
Please use tf.data to implement this functionality.
Extracting MNIST\_data/train-labels-idx1-ubyte.gz
WARNING:tensorflow:From /Users/matto/anaconda3/lib/python3.6/site-packages/tensorflow/contrib/learn/python/learn/datasets/mnist.py:110: dense\_to\_one\_hot (from tensorflow.contrib.learn.python.learn.datasets.mnist) is deprecated and will be removed in a future version.
Instructions for updating:
Please use tf.one\_hot on tensors.
Extracting MNIST\_data/t10k-images-idx3-ubyte.gz
Extracting MNIST\_data/t10k-labels-idx1-ubyte.gz
WARNING:tensorflow:From /Users/matto/anaconda3/lib/python3.6/site-packages/tensorflow/contrib/learn/python/learn/datasets/mnist.py:290: DataSet.\_\_init\_\_ (from tensorflow.contrib.learn.python.learn.datasets.mnist) is deprecated and will be removed in a future version.
Instructions for updating:
Please use alternatives such as official/mnist/dataset.py from tensorflow/models.

    \end{Verbatim}

    \begin{Verbatim}[commandchars=\\\{\}]
{\color{incolor}In [{\color{incolor}20}]:} \PY{k+kn}{import} \PY{n+nn}{tensorflow} \PY{k}{as} \PY{n+nn}{tf}
         \PY{n}{sess} \PY{o}{=} \PY{n}{tf}\PY{o}{.}\PY{n}{InteractiveSession}\PY{p}{(}\PY{p}{)}
\end{Verbatim}


    \begin{Verbatim}[commandchars=\\\{\}]
/Users/matto/anaconda3/lib/python3.6/site-packages/tensorflow/python/client/session.py:1711: UserWarning: An interactive session is already active. This can cause out-of-memory errors in some cases. You must explicitly call `InteractiveSession.close()` to release resources held by the other session(s).
  warnings.warn('An interactive session is already active. This can '

    \end{Verbatim}

    \begin{Verbatim}[commandchars=\\\{\}]
{\color{incolor}In [{\color{incolor}21}]:} \PY{n}{x} \PY{o}{=} \PY{n}{tf}\PY{o}{.}\PY{n}{placeholder}\PY{p}{(}\PY{n}{tf}\PY{o}{.}\PY{n}{float32}\PY{p}{,} \PY{n}{shape}\PY{o}{=}\PY{p}{[}\PY{k+kc}{None}\PY{p}{,} \PY{l+m+mi}{784}\PY{p}{]}\PY{p}{)}
         \PY{n}{y\PYZus{}} \PY{o}{=} \PY{n}{tf}\PY{o}{.}\PY{n}{placeholder}\PY{p}{(}\PY{n}{tf}\PY{o}{.}\PY{n}{float32}\PY{p}{,} \PY{n}{shape}\PY{o}{=}\PY{p}{[}\PY{k+kc}{None}\PY{p}{,} \PY{l+m+mi}{10}\PY{p}{]}\PY{p}{)}
\end{Verbatim}


    \begin{Verbatim}[commandchars=\\\{\}]
{\color{incolor}In [{\color{incolor}22}]:} \PY{n}{W} \PY{o}{=} \PY{n}{tf}\PY{o}{.}\PY{n}{Variable}\PY{p}{(}\PY{n}{tf}\PY{o}{.}\PY{n}{zeros}\PY{p}{(}\PY{p}{[}\PY{l+m+mi}{784}\PY{p}{,}\PY{l+m+mi}{10}\PY{p}{]}\PY{p}{)}\PY{p}{)}
         \PY{n}{b} \PY{o}{=} \PY{n}{tf}\PY{o}{.}\PY{n}{Variable}\PY{p}{(}\PY{n}{tf}\PY{o}{.}\PY{n}{zeros}\PY{p}{(}\PY{p}{[}\PY{l+m+mi}{10}\PY{p}{]}\PY{p}{)}\PY{p}{)}
\end{Verbatim}


    \begin{Verbatim}[commandchars=\\\{\}]
{\color{incolor}In [{\color{incolor}23}]:} \PY{n}{sess}\PY{o}{.}\PY{n}{run}\PY{p}{(}\PY{n}{tf}\PY{o}{.}\PY{n}{global\PYZus{}variables\PYZus{}initializer}\PY{p}{(}\PY{p}{)}\PY{p}{)}
\end{Verbatim}


    \begin{Verbatim}[commandchars=\\\{\}]
{\color{incolor}In [{\color{incolor}24}]:} \PY{n}{y} \PY{o}{=} \PY{n}{tf}\PY{o}{.}\PY{n}{matmul}\PY{p}{(}\PY{n}{x}\PY{p}{,}\PY{n}{W}\PY{p}{)} \PY{o}{+} \PY{n}{b}
\end{Verbatim}


    \begin{Verbatim}[commandchars=\\\{\}]
{\color{incolor}In [{\color{incolor}25}]:} \PY{n}{cross\PYZus{}entropy} \PY{o}{=} \PY{n}{tf}\PY{o}{.}\PY{n}{reduce\PYZus{}mean}\PY{p}{(}
             \PY{n}{tf}\PY{o}{.}\PY{n}{nn}\PY{o}{.}\PY{n}{softmax\PYZus{}cross\PYZus{}entropy\PYZus{}with\PYZus{}logits}\PY{p}{(}\PY{n}{labels}\PY{o}{=}\PY{n}{y\PYZus{}}\PY{p}{,} \PY{n}{logits}\PY{o}{=}\PY{n}{y}\PY{p}{)}\PY{p}{)}
\end{Verbatim}


    \begin{Verbatim}[commandchars=\\\{\}]
{\color{incolor}In [{\color{incolor}26}]:} \PY{k}{def} \PY{n+nf}{weight\PYZus{}variable}\PY{p}{(}\PY{n}{shape}\PY{p}{)}\PY{p}{:}
           \PY{n}{initial} \PY{o}{=} \PY{n}{tf}\PY{o}{.}\PY{n}{truncated\PYZus{}normal}\PY{p}{(}\PY{n}{shape}\PY{p}{,} \PY{n}{stddev}\PY{o}{=}\PY{l+m+mf}{0.1}\PY{p}{)}
           \PY{k}{return} \PY{n}{tf}\PY{o}{.}\PY{n}{Variable}\PY{p}{(}\PY{n}{initial}\PY{p}{)}
         
         \PY{c+c1}{\PYZsh{} SoftMaxの代わりにReLUを使う場合、Biasはすこし与えたほうが良い。}
         \PY{k}{def} \PY{n+nf}{bias\PYZus{}variable}\PY{p}{(}\PY{n}{shape}\PY{p}{)}\PY{p}{:}
           \PY{n}{initial} \PY{o}{=} \PY{n}{tf}\PY{o}{.}\PY{n}{constant}\PY{p}{(}\PY{l+m+mf}{0.1}\PY{p}{,} \PY{n}{shape}\PY{o}{=}\PY{n}{shape}\PY{p}{)}
           \PY{k}{return} \PY{n}{tf}\PY{o}{.}\PY{n}{Variable}\PY{p}{(}\PY{n}{initial}\PY{p}{)}
\end{Verbatim}


    \begin{Verbatim}[commandchars=\\\{\}]
{\color{incolor}In [{\color{incolor}27}]:} \PY{k}{def} \PY{n+nf}{conv2d}\PY{p}{(}\PY{n}{x}\PY{p}{,} \PY{n}{W}\PY{p}{)}\PY{p}{:}
           \PY{k}{return} \PY{n}{tf}\PY{o}{.}\PY{n}{nn}\PY{o}{.}\PY{n}{conv2d}\PY{p}{(}\PY{n}{x}\PY{p}{,} \PY{n}{W}\PY{p}{,} \PY{n}{strides}\PY{o}{=}\PY{p}{[}\PY{l+m+mi}{1}\PY{p}{,} \PY{l+m+mi}{1}\PY{p}{,} \PY{l+m+mi}{1}\PY{p}{,} \PY{l+m+mi}{1}\PY{p}{]}\PY{p}{,} \PY{n}{padding}\PY{o}{=}\PY{l+s+s1}{\PYZsq{}}\PY{l+s+s1}{SAME}\PY{l+s+s1}{\PYZsq{}}\PY{p}{)}
         
         \PY{k}{def} \PY{n+nf}{max\PYZus{}pool\PYZus{}2x2}\PY{p}{(}\PY{n}{x}\PY{p}{)}\PY{p}{:}
           \PY{k}{return} \PY{n}{tf}\PY{o}{.}\PY{n}{nn}\PY{o}{.}\PY{n}{max\PYZus{}pool}\PY{p}{(}\PY{n}{x}\PY{p}{,} \PY{n}{ksize}\PY{o}{=}\PY{p}{[}\PY{l+m+mi}{1}\PY{p}{,} \PY{l+m+mi}{2}\PY{p}{,} \PY{l+m+mi}{2}\PY{p}{,} \PY{l+m+mi}{1}\PY{p}{]}\PY{p}{,}
                                 \PY{n}{strides}\PY{o}{=}\PY{p}{[}\PY{l+m+mi}{1}\PY{p}{,} \PY{l+m+mi}{2}\PY{p}{,} \PY{l+m+mi}{2}\PY{p}{,} \PY{l+m+mi}{1}\PY{p}{]}\PY{p}{,} \PY{n}{padding}\PY{o}{=}\PY{l+s+s1}{\PYZsq{}}\PY{l+s+s1}{SAME}\PY{l+s+s1}{\PYZsq{}}\PY{p}{)}
\end{Verbatim}


    \begin{Verbatim}[commandchars=\\\{\}]
{\color{incolor}In [{\color{incolor}28}]:} \PY{n}{W\PYZus{}conv1} \PY{o}{=} \PY{n}{weight\PYZus{}variable}\PY{p}{(}\PY{p}{[}\PY{l+m+mi}{5}\PY{p}{,} \PY{l+m+mi}{5}\PY{p}{,} \PY{l+m+mi}{1}\PY{p}{,} \PY{l+m+mi}{32}\PY{p}{]}\PY{p}{)}
         \PY{n}{b\PYZus{}conv1} \PY{o}{=} \PY{n}{bias\PYZus{}variable}\PY{p}{(}\PY{p}{[}\PY{l+m+mi}{32}\PY{p}{]}\PY{p}{)}
\end{Verbatim}


    \begin{Verbatim}[commandchars=\\\{\}]
{\color{incolor}In [{\color{incolor}29}]:} \PY{n}{x\PYZus{}image} \PY{o}{=} \PY{n}{tf}\PY{o}{.}\PY{n}{reshape}\PY{p}{(}\PY{n}{x}\PY{p}{,} \PY{p}{[}\PY{o}{\PYZhy{}}\PY{l+m+mi}{1}\PY{p}{,}\PY{l+m+mi}{28}\PY{p}{,}\PY{l+m+mi}{28}\PY{p}{,}\PY{l+m+mi}{1}\PY{p}{]}\PY{p}{)}
\end{Verbatim}


    \begin{Verbatim}[commandchars=\\\{\}]
{\color{incolor}In [{\color{incolor}30}]:} \PY{n}{h\PYZus{}conv1} \PY{o}{=} \PY{n}{tf}\PY{o}{.}\PY{n}{nn}\PY{o}{.}\PY{n}{relu}\PY{p}{(}\PY{n}{conv2d}\PY{p}{(}\PY{n}{x\PYZus{}image}\PY{p}{,} \PY{n}{W\PYZus{}conv1}\PY{p}{)} \PY{o}{+} \PY{n}{b\PYZus{}conv1}\PY{p}{)}
         \PY{n}{h\PYZus{}pool1} \PY{o}{=} \PY{n}{max\PYZus{}pool\PYZus{}2x2}\PY{p}{(}\PY{n}{h\PYZus{}conv1}\PY{p}{)}
\end{Verbatim}


    \begin{Verbatim}[commandchars=\\\{\}]
{\color{incolor}In [{\color{incolor}31}]:} \PY{n}{W\PYZus{}conv2} \PY{o}{=} \PY{n}{weight\PYZus{}variable}\PY{p}{(}\PY{p}{[}\PY{l+m+mi}{5}\PY{p}{,} \PY{l+m+mi}{5}\PY{p}{,} \PY{l+m+mi}{32}\PY{p}{,} \PY{l+m+mi}{64}\PY{p}{]}\PY{p}{)}
         \PY{n}{b\PYZus{}conv2} \PY{o}{=} \PY{n}{bias\PYZus{}variable}\PY{p}{(}\PY{p}{[}\PY{l+m+mi}{64}\PY{p}{]}\PY{p}{)}
         
         \PY{n}{h\PYZus{}conv2} \PY{o}{=} \PY{n}{tf}\PY{o}{.}\PY{n}{nn}\PY{o}{.}\PY{n}{relu}\PY{p}{(}\PY{n}{conv2d}\PY{p}{(}\PY{n}{h\PYZus{}pool1}\PY{p}{,} \PY{n}{W\PYZus{}conv2}\PY{p}{)} \PY{o}{+} \PY{n}{b\PYZus{}conv2}\PY{p}{)}
         \PY{n}{h\PYZus{}pool2} \PY{o}{=} \PY{n}{max\PYZus{}pool\PYZus{}2x2}\PY{p}{(}\PY{n}{h\PYZus{}conv2}\PY{p}{)}
\end{Verbatim}


    \begin{Verbatim}[commandchars=\\\{\}]
{\color{incolor}In [{\color{incolor}32}]:} \PY{n}{W\PYZus{}fc1} \PY{o}{=} \PY{n}{weight\PYZus{}variable}\PY{p}{(}\PY{p}{[}\PY{l+m+mi}{7} \PY{o}{*} \PY{l+m+mi}{7} \PY{o}{*} \PY{l+m+mi}{64}\PY{p}{,} \PY{l+m+mi}{1024}\PY{p}{]}\PY{p}{)}
         \PY{n}{b\PYZus{}fc1} \PY{o}{=} \PY{n}{bias\PYZus{}variable}\PY{p}{(}\PY{p}{[}\PY{l+m+mi}{1024}\PY{p}{]}\PY{p}{)}
         
         \PY{n}{h\PYZus{}pool2\PYZus{}flat} \PY{o}{=} \PY{n}{tf}\PY{o}{.}\PY{n}{reshape}\PY{p}{(}\PY{n}{h\PYZus{}pool2}\PY{p}{,} \PY{p}{[}\PY{o}{\PYZhy{}}\PY{l+m+mi}{1}\PY{p}{,} \PY{l+m+mi}{7}\PY{o}{*}\PY{l+m+mi}{7}\PY{o}{*}\PY{l+m+mi}{64}\PY{p}{]}\PY{p}{)}
         \PY{n}{h\PYZus{}fc1} \PY{o}{=} \PY{n}{tf}\PY{o}{.}\PY{n}{nn}\PY{o}{.}\PY{n}{relu}\PY{p}{(}\PY{n}{tf}\PY{o}{.}\PY{n}{matmul}\PY{p}{(}\PY{n}{h\PYZus{}pool2\PYZus{}flat}\PY{p}{,} \PY{n}{W\PYZus{}fc1}\PY{p}{)} \PY{o}{+} \PY{n}{b\PYZus{}fc1}\PY{p}{)}
\end{Verbatim}


    \begin{Verbatim}[commandchars=\\\{\}]
{\color{incolor}In [{\color{incolor}33}]:} \PY{n}{keep\PYZus{}prob} \PY{o}{=} \PY{n}{tf}\PY{o}{.}\PY{n}{placeholder}\PY{p}{(}\PY{n}{tf}\PY{o}{.}\PY{n}{float32}\PY{p}{)}
         \PY{n}{h\PYZus{}fc1\PYZus{}drop} \PY{o}{=} \PY{n}{tf}\PY{o}{.}\PY{n}{nn}\PY{o}{.}\PY{n}{dropout}\PY{p}{(}\PY{n}{h\PYZus{}fc1}\PY{p}{,} \PY{n}{keep\PYZus{}prob}\PY{p}{)}
\end{Verbatim}


    \begin{Verbatim}[commandchars=\\\{\}]
{\color{incolor}In [{\color{incolor}34}]:} \PY{n}{W\PYZus{}fc2} \PY{o}{=} \PY{n}{weight\PYZus{}variable}\PY{p}{(}\PY{p}{[}\PY{l+m+mi}{1024}\PY{p}{,} \PY{l+m+mi}{10}\PY{p}{]}\PY{p}{)}
         \PY{n}{b\PYZus{}fc2} \PY{o}{=} \PY{n}{bias\PYZus{}variable}\PY{p}{(}\PY{p}{[}\PY{l+m+mi}{10}\PY{p}{]}\PY{p}{)}
         
         \PY{n}{y\PYZus{}conv} \PY{o}{=} \PY{n}{tf}\PY{o}{.}\PY{n}{matmul}\PY{p}{(}\PY{n}{h\PYZus{}fc1\PYZus{}drop}\PY{p}{,} \PY{n}{W\PYZus{}fc2}\PY{p}{)} \PY{o}{+} \PY{n}{b\PYZus{}fc2}
\end{Verbatim}


    \begin{Verbatim}[commandchars=\\\{\}]
{\color{incolor}In [{\color{incolor}35}]:} \PY{n}{cross\PYZus{}entropy} \PY{o}{=} \PY{n}{tf}\PY{o}{.}\PY{n}{reduce\PYZus{}mean}\PY{p}{(}
             \PY{n}{tf}\PY{o}{.}\PY{n}{nn}\PY{o}{.}\PY{n}{softmax\PYZus{}cross\PYZus{}entropy\PYZus{}with\PYZus{}logits}\PY{p}{(}\PY{n}{labels}\PY{o}{=}\PY{n}{y\PYZus{}}\PY{p}{,}
                                                     \PY{n}{logits}\PY{o}{=}\PY{n}{y\PYZus{}conv}\PY{p}{)}\PY{p}{)}
         \PY{n}{train\PYZus{}step} \PY{o}{=} \PY{n}{tf}\PY{o}{.}\PY{n}{train}\PY{o}{.}\PY{n}{AdamOptimizer}\PY{p}{(}\PY{l+m+mf}{1e\PYZhy{}4}\PY{p}{)}\PY{o}{.}\PY{n}{minimize}\PY{p}{(}\PY{n}{cross\PYZus{}entropy}\PY{p}{)}
         \PY{n}{correct\PYZus{}prediction} \PY{o}{=} \PY{n}{tf}\PY{o}{.}\PY{n}{equal}\PY{p}{(}\PY{n}{tf}\PY{o}{.}\PY{n}{argmax}\PY{p}{(}\PY{n}{y\PYZus{}conv}\PY{p}{,}\PY{l+m+mi}{1}\PY{p}{)}\PY{p}{,} \PY{n}{tf}\PY{o}{.}\PY{n}{argmax}\PY{p}{(}\PY{n}{y\PYZus{}}\PY{p}{,}\PY{l+m+mi}{1}\PY{p}{)}\PY{p}{)}
         \PY{n}{accuracy} \PY{o}{=} \PY{n}{tf}\PY{o}{.}\PY{n}{reduce\PYZus{}mean}\PY{p}{(}\PY{n}{tf}\PY{o}{.}\PY{n}{cast}\PY{p}{(}\PY{n}{correct\PYZus{}prediction}\PY{p}{,} \PY{n}{tf}\PY{o}{.}\PY{n}{float32}\PY{p}{)}\PY{p}{)}
         \PY{n}{sess}\PY{o}{.}\PY{n}{run}\PY{p}{(}\PY{n}{tf}\PY{o}{.}\PY{n}{global\PYZus{}variables\PYZus{}initializer}\PY{p}{(}\PY{p}{)}\PY{p}{)}
         
         \PY{k}{for} \PY{n}{i} \PY{o+ow}{in} \PY{n+nb}{range}\PY{p}{(}\PY{l+m+mi}{20000}\PY{p}{)}\PY{p}{:}
           \PY{n}{batch} \PY{o}{=} \PY{n}{mnist}\PY{o}{.}\PY{n}{train}\PY{o}{.}\PY{n}{next\PYZus{}batch}\PY{p}{(}\PY{l+m+mi}{50}\PY{p}{)}
           \PY{k}{if} \PY{n}{i}\PY{o}{\PYZpc{}}\PY{k}{100} == 0:
             \PY{n}{train\PYZus{}accuracy} \PY{o}{=} \PY{n}{accuracy}\PY{o}{.}\PY{n}{eval}\PY{p}{(}\PY{n}{feed\PYZus{}dict}\PY{o}{=}\PY{p}{\PYZob{}}
                 \PY{n}{x}\PY{p}{:}\PY{n}{batch}\PY{p}{[}\PY{l+m+mi}{0}\PY{p}{]}\PY{p}{,} \PY{n}{y\PYZus{}}\PY{p}{:} \PY{n}{batch}\PY{p}{[}\PY{l+m+mi}{1}\PY{p}{]}\PY{p}{,} \PY{n}{keep\PYZus{}prob}\PY{p}{:} \PY{l+m+mf}{1.0}\PY{p}{\PYZcb{}}\PY{p}{)}
             \PY{n+nb}{print}\PY{p}{(}\PY{l+s+s2}{\PYZdq{}}\PY{l+s+s2}{step }\PY{l+s+si}{\PYZpc{}d}\PY{l+s+s2}{, training accuracy }\PY{l+s+si}{\PYZpc{}g}\PY{l+s+s2}{\PYZdq{}}\PY{o}{\PYZpc{}}\PY{p}{(}\PY{n}{i}\PY{p}{,} \PY{n}{train\PYZus{}accuracy}\PY{p}{)}\PY{p}{)}
           \PY{k}{if} \PY{n}{i}\PY{o}{\PYZpc{}}\PY{k}{1000} == 0:
             \PY{n+nb}{print}\PY{p}{(}\PY{l+s+s2}{\PYZdq{}}\PY{l+s+s2}{test accuracy }\PY{l+s+si}{\PYZpc{}g}\PY{l+s+s2}{\PYZdq{}}\PY{o}{\PYZpc{}}\PY{k}{accuracy}.eval(feed\PYZus{}dict=\PYZob{}
               \PY{n}{x}\PY{p}{:} \PY{n}{mnist}\PY{o}{.}\PY{n}{test}\PY{o}{.}\PY{n}{images}\PY{p}{,} \PY{n}{y\PYZus{}}\PY{p}{:} \PY{n}{mnist}\PY{o}{.}\PY{n}{test}\PY{o}{.}\PY{n}{labels}\PY{p}{,} \PY{n}{keep\PYZus{}prob}\PY{p}{:} \PY{l+m+mf}{1.0}\PY{p}{\PYZcb{}}\PY{p}{)}\PY{p}{)}  
           \PY{n}{train\PYZus{}step}\PY{o}{.}\PY{n}{run}\PY{p}{(}\PY{n}{feed\PYZus{}dict}\PY{o}{=}\PY{p}{\PYZob{}}\PY{n}{x}\PY{p}{:} \PY{n}{batch}\PY{p}{[}\PY{l+m+mi}{0}\PY{p}{]}\PY{p}{,} \PY{n}{y\PYZus{}}\PY{p}{:} \PY{n}{batch}\PY{p}{[}\PY{l+m+mi}{1}\PY{p}{]}\PY{p}{,} \PY{n}{keep\PYZus{}prob}\PY{p}{:} \PY{l+m+mf}{0.5}\PY{p}{\PYZcb{}}\PY{p}{)}
         
         \PY{n+nb}{print}\PY{p}{(}\PY{l+s+s2}{\PYZdq{}}\PY{l+s+s2}{test accuracy }\PY{l+s+si}{\PYZpc{}g}\PY{l+s+s2}{\PYZdq{}}\PY{o}{\PYZpc{}}\PY{k}{accuracy}.eval(feed\PYZus{}dict=\PYZob{}
               \PY{n}{x}\PY{p}{:} \PY{n}{mnist}\PY{o}{.}\PY{n}{test}\PY{o}{.}\PY{n}{images}\PY{p}{,} \PY{n}{y\PYZus{}}\PY{p}{:} \PY{n}{mnist}\PY{o}{.}\PY{n}{test}\PY{o}{.}\PY{n}{labels}\PY{p}{,} \PY{n}{keep\PYZus{}prob}\PY{p}{:} \PY{l+m+mf}{1.0}\PY{p}{\PYZcb{}}\PY{p}{)}\PY{p}{)}  
\end{Verbatim}


    \begin{Verbatim}[commandchars=\\\{\}]
step 0, training accuracy 0.08
test accuracy 0.0925
step 100, training accuracy 0.82
step 200, training accuracy 0.84
step 300, training accuracy 0.94
step 400, training accuracy 0.96
step 500, training accuracy 0.92
step 600, training accuracy 0.94
step 700, training accuracy 0.98
step 800, training accuracy 0.98
step 900, training accuracy 1
step 1000, training accuracy 0.98
test accuracy 0.9628
step 1100, training accuracy 0.92
step 1200, training accuracy 0.94
step 1300, training accuracy 1
step 1400, training accuracy 1
step 1500, training accuracy 0.9
step 1600, training accuracy 1
step 1700, training accuracy 0.98
step 1800, training accuracy 0.96
step 1900, training accuracy 0.98
step 2000, training accuracy 1
test accuracy 0.9764
step 2100, training accuracy 0.98
step 2200, training accuracy 1
step 2300, training accuracy 0.98
step 2400, training accuracy 0.96
step 2500, training accuracy 0.92
step 2600, training accuracy 1
step 2700, training accuracy 1
step 2800, training accuracy 0.98
step 2900, training accuracy 0.98
step 3000, training accuracy 1
test accuracy 0.9808
step 3100, training accuracy 1
step 3200, training accuracy 1
step 3300, training accuracy 0.98
step 3400, training accuracy 1
step 3500, training accuracy 1
step 3600, training accuracy 0.96
step 3700, training accuracy 0.98
step 3800, training accuracy 0.98
step 3900, training accuracy 0.98
step 4000, training accuracy 0.96
test accuracy 0.9846
step 4100, training accuracy 0.96
step 4200, training accuracy 1
step 4300, training accuracy 1
step 4400, training accuracy 0.98
step 4500, training accuracy 1
step 4600, training accuracy 1
step 4700, training accuracy 0.96
step 4800, training accuracy 1
step 4900, training accuracy 0.94
step 5000, training accuracy 1
test accuracy 0.9858
step 5100, training accuracy 1
step 5200, training accuracy 0.96
step 5300, training accuracy 1
step 5400, training accuracy 1
step 5500, training accuracy 0.98
step 5600, training accuracy 0.98
step 5700, training accuracy 1
step 5800, training accuracy 1
step 5900, training accuracy 1
step 6000, training accuracy 0.98
test accuracy 0.9879
step 6100, training accuracy 1
step 6200, training accuracy 1
step 6300, training accuracy 1
step 6400, training accuracy 1
step 6500, training accuracy 1
step 6600, training accuracy 0.98
step 6700, training accuracy 1
step 6800, training accuracy 1
step 6900, training accuracy 0.98
step 7000, training accuracy 1
test accuracy 0.9883
step 7100, training accuracy 0.96
step 7200, training accuracy 1
step 7300, training accuracy 1
step 7400, training accuracy 0.98
step 7500, training accuracy 1
step 7600, training accuracy 1
step 7700, training accuracy 1
step 7800, training accuracy 1
step 7900, training accuracy 1
step 8000, training accuracy 1
test accuracy 0.9897
step 8100, training accuracy 0.98
step 8200, training accuracy 1
step 8300, training accuracy 0.96
step 8400, training accuracy 1
step 8500, training accuracy 1
step 8600, training accuracy 0.98
step 8700, training accuracy 1
step 8800, training accuracy 1
step 8900, training accuracy 0.98
step 9000, training accuracy 1
test accuracy 0.9901
step 9100, training accuracy 0.98
step 9200, training accuracy 1
step 9300, training accuracy 1
step 9400, training accuracy 0.98
step 9500, training accuracy 1
step 9600, training accuracy 1
step 9700, training accuracy 1
step 9800, training accuracy 1
step 9900, training accuracy 1
step 10000, training accuracy 1
test accuracy 0.9893
step 10100, training accuracy 1
step 10200, training accuracy 1
step 10300, training accuracy 0.98
step 10400, training accuracy 1
step 10500, training accuracy 1
step 10600, training accuracy 1
step 10700, training accuracy 1
step 10800, training accuracy 1
step 10900, training accuracy 1
step 11000, training accuracy 1
test accuracy 0.9902
step 11100, training accuracy 1
step 11200, training accuracy 1
step 11300, training accuracy 1
step 11400, training accuracy 1
step 11500, training accuracy 1
step 11600, training accuracy 1
step 11700, training accuracy 0.98
step 11800, training accuracy 1
step 11900, training accuracy 1
step 12000, training accuracy 0.98
test accuracy 0.9898
step 12100, training accuracy 1
step 12200, training accuracy 1
step 12300, training accuracy 1
step 12400, training accuracy 1
step 12500, training accuracy 1
step 12600, training accuracy 1
step 12700, training accuracy 1
step 12800, training accuracy 1
step 12900, training accuracy 1
step 13000, training accuracy 1
test accuracy 0.9912
step 13100, training accuracy 1
step 13200, training accuracy 1
step 13300, training accuracy 1
step 13400, training accuracy 1
step 13500, training accuracy 0.98
step 13600, training accuracy 1
step 13700, training accuracy 1
step 13800, training accuracy 1
step 13900, training accuracy 1
step 14000, training accuracy 1
test accuracy 0.9918
step 14100, training accuracy 1
step 14200, training accuracy 1
step 14300, training accuracy 1
step 14400, training accuracy 1
step 14500, training accuracy 1
step 14600, training accuracy 1
step 14700, training accuracy 1
step 14800, training accuracy 1
step 14900, training accuracy 1
step 15000, training accuracy 1
test accuracy 0.992
step 15100, training accuracy 1
step 15200, training accuracy 1
step 15300, training accuracy 1
step 15400, training accuracy 1
step 15500, training accuracy 1
step 15600, training accuracy 1
step 15700, training accuracy 1
step 15800, training accuracy 0.98
step 15900, training accuracy 1
step 16000, training accuracy 1
test accuracy 0.9912
step 16100, training accuracy 1
step 16200, training accuracy 1
step 16300, training accuracy 1
step 16400, training accuracy 1
step 16500, training accuracy 1
step 16600, training accuracy 1
step 16700, training accuracy 1
step 16800, training accuracy 1
step 16900, training accuracy 1
step 17000, training accuracy 1
test accuracy 0.9921
step 17100, training accuracy 1
step 17200, training accuracy 1
step 17300, training accuracy 1
step 17400, training accuracy 1
step 17500, training accuracy 1
step 17600, training accuracy 1
step 17700, training accuracy 1
step 17800, training accuracy 1
step 17900, training accuracy 1
step 18000, training accuracy 1
test accuracy 0.9919
step 18100, training accuracy 1
step 18200, training accuracy 1
step 18300, training accuracy 1
step 18400, training accuracy 1
step 18500, training accuracy 1
step 18600, training accuracy 1
step 18700, training accuracy 1
step 18800, training accuracy 1
step 18900, training accuracy 1
step 19000, training accuracy 1
test accuracy 0.9931
step 19100, training accuracy 1
step 19200, training accuracy 1
step 19300, training accuracy 1
step 19400, training accuracy 1
step 19500, training accuracy 1
step 19600, training accuracy 1
step 19700, training accuracy 1
step 19800, training accuracy 1
step 19900, training accuracy 1

    \end{Verbatim}

    TensorFlowをそのまま使うとあまり読みやすくないので、Kerasでwrapしてみる。

\begin{itemize}
\tightlist
\item
  https://qiita.com/cvusk/items/2cd3e516276b426bc58c
\end{itemize}

    \begin{Verbatim}[commandchars=\\\{\}]
{\color{incolor}In [{\color{incolor}23}]:} \PY{k+kn}{import} \PY{n+nn}{keras}
         \PY{k+kn}{from} \PY{n+nn}{keras}\PY{n+nn}{.}\PY{n+nn}{datasets} \PY{k}{import} \PY{n}{mnist}
         \PY{k+kn}{from} \PY{n+nn}{keras}\PY{n+nn}{.}\PY{n+nn}{models} \PY{k}{import} \PY{n}{Sequential}
         \PY{k+kn}{from} \PY{n+nn}{keras}\PY{n+nn}{.}\PY{n+nn}{layers} \PY{k}{import} \PY{n}{Dense}\PY{p}{,} \PY{n}{Dropout}
         \PY{k+kn}{from} \PY{n+nn}{keras}\PY{n+nn}{.}\PY{n+nn}{optimizers} \PY{k}{import} \PY{n}{RMSprop}
         
         \PY{n}{batch\PYZus{}size} \PY{o}{=} \PY{l+m+mi}{128}
         \PY{n}{num\PYZus{}classes} \PY{o}{=} \PY{l+m+mi}{10}
         \PY{n}{epochs} \PY{o}{=} \PY{l+m+mi}{20}
         
         \PY{c+c1}{\PYZsh{} the data, shuffled and split between train and test sets}
         \PY{p}{(}\PY{n}{x\PYZus{}train}\PY{p}{,} \PY{n}{y\PYZus{}train}\PY{p}{)}\PY{p}{,} \PY{p}{(}\PY{n}{x\PYZus{}test}\PY{p}{,} \PY{n}{y\PYZus{}test}\PY{p}{)} \PY{o}{=} \PY{n}{mnist}\PY{o}{.}\PY{n}{load\PYZus{}data}\PY{p}{(}\PY{p}{)}
         
         \PY{n}{x\PYZus{}train} \PY{o}{=} \PY{n}{x\PYZus{}train}\PY{o}{.}\PY{n}{reshape}\PY{p}{(}\PY{l+m+mi}{60000}\PY{p}{,} \PY{l+m+mi}{784}\PY{p}{)}
         \PY{n}{x\PYZus{}test} \PY{o}{=} \PY{n}{x\PYZus{}test}\PY{o}{.}\PY{n}{reshape}\PY{p}{(}\PY{l+m+mi}{10000}\PY{p}{,} \PY{l+m+mi}{784}\PY{p}{)}
         \PY{n}{x\PYZus{}train} \PY{o}{=} \PY{n}{x\PYZus{}train}\PY{o}{.}\PY{n}{astype}\PY{p}{(}\PY{l+s+s1}{\PYZsq{}}\PY{l+s+s1}{float32}\PY{l+s+s1}{\PYZsq{}}\PY{p}{)}
         \PY{n}{x\PYZus{}test} \PY{o}{=} \PY{n}{x\PYZus{}test}\PY{o}{.}\PY{n}{astype}\PY{p}{(}\PY{l+s+s1}{\PYZsq{}}\PY{l+s+s1}{float32}\PY{l+s+s1}{\PYZsq{}}\PY{p}{)}
         \PY{n}{x\PYZus{}train} \PY{o}{/}\PY{o}{=} \PY{l+m+mi}{255}
         \PY{n}{x\PYZus{}test} \PY{o}{/}\PY{o}{=} \PY{l+m+mi}{255}
         \PY{n+nb}{print}\PY{p}{(}\PY{n}{x\PYZus{}train}\PY{o}{.}\PY{n}{shape}\PY{p}{[}\PY{l+m+mi}{0}\PY{p}{]}\PY{p}{,} \PY{l+s+s1}{\PYZsq{}}\PY{l+s+s1}{train samples}\PY{l+s+s1}{\PYZsq{}}\PY{p}{)}
         \PY{n+nb}{print}\PY{p}{(}\PY{n}{x\PYZus{}test}\PY{o}{.}\PY{n}{shape}\PY{p}{[}\PY{l+m+mi}{0}\PY{p}{]}\PY{p}{,} \PY{l+s+s1}{\PYZsq{}}\PY{l+s+s1}{test samples}\PY{l+s+s1}{\PYZsq{}}\PY{p}{)}
         
         \PY{c+c1}{\PYZsh{} convert class vectors to binary class matrices}
         \PY{n}{y\PYZus{}train} \PY{o}{=} \PY{n}{keras}\PY{o}{.}\PY{n}{utils}\PY{o}{.}\PY{n}{to\PYZus{}categorical}\PY{p}{(}\PY{n}{y\PYZus{}train}\PY{p}{,} \PY{n}{num\PYZus{}classes}\PY{p}{)}
         \PY{n}{y\PYZus{}test} \PY{o}{=} \PY{n}{keras}\PY{o}{.}\PY{n}{utils}\PY{o}{.}\PY{n}{to\PYZus{}categorical}\PY{p}{(}\PY{n}{y\PYZus{}test}\PY{p}{,} \PY{n}{num\PYZus{}classes}\PY{p}{)}
\end{Verbatim}


    \begin{Verbatim}[commandchars=\\\{\}]
Using TensorFlow backend.

    \end{Verbatim}

    \begin{Verbatim}[commandchars=\\\{\}]
Downloading data from https://s3.amazonaws.com/img-datasets/mnist.npz
11493376/11490434 [==============================] - 5s 0us/step
60000 train samples
10000 test samples

    \end{Verbatim}

    \begin{Verbatim}[commandchars=\\\{\}]
{\color{incolor}In [{\color{incolor}24}]:} \PY{n}{model} \PY{o}{=} \PY{n}{Sequential}\PY{p}{(}\PY{p}{)}
         \PY{n}{model}\PY{o}{.}\PY{n}{add}\PY{p}{(}\PY{n}{Dense}\PY{p}{(}\PY{l+m+mi}{512}\PY{p}{,} \PY{n}{activation}\PY{o}{=}\PY{l+s+s1}{\PYZsq{}}\PY{l+s+s1}{relu}\PY{l+s+s1}{\PYZsq{}}\PY{p}{,} \PY{n}{input\PYZus{}shape}\PY{o}{=}\PY{p}{(}\PY{l+m+mi}{784}\PY{p}{,}\PY{p}{)}\PY{p}{)}\PY{p}{)}
         \PY{n}{model}\PY{o}{.}\PY{n}{add}\PY{p}{(}\PY{n}{Dropout}\PY{p}{(}\PY{l+m+mf}{0.2}\PY{p}{)}\PY{p}{)}
         \PY{n}{model}\PY{o}{.}\PY{n}{add}\PY{p}{(}\PY{n}{Dense}\PY{p}{(}\PY{l+m+mi}{512}\PY{p}{,} \PY{n}{activation}\PY{o}{=}\PY{l+s+s1}{\PYZsq{}}\PY{l+s+s1}{relu}\PY{l+s+s1}{\PYZsq{}}\PY{p}{)}\PY{p}{)}
         \PY{n}{model}\PY{o}{.}\PY{n}{add}\PY{p}{(}\PY{n}{Dropout}\PY{p}{(}\PY{l+m+mf}{0.2}\PY{p}{)}\PY{p}{)}
         \PY{n}{model}\PY{o}{.}\PY{n}{add}\PY{p}{(}\PY{n}{Dense}\PY{p}{(}\PY{n}{num\PYZus{}classes}\PY{p}{,} \PY{n}{activation}\PY{o}{=}\PY{l+s+s1}{\PYZsq{}}\PY{l+s+s1}{softmax}\PY{l+s+s1}{\PYZsq{}}\PY{p}{)}\PY{p}{)}
         
         \PY{n}{model}\PY{o}{.}\PY{n}{summary}\PY{p}{(}\PY{p}{)}
         
         \PY{c+c1}{\PYZsh{} 隠れ層が2枚あるNN?}
\end{Verbatim}


    \begin{Verbatim}[commandchars=\\\{\}]
\_\_\_\_\_\_\_\_\_\_\_\_\_\_\_\_\_\_\_\_\_\_\_\_\_\_\_\_\_\_\_\_\_\_\_\_\_\_\_\_\_\_\_\_\_\_\_\_\_\_\_\_\_\_\_\_\_\_\_\_\_\_\_\_\_
Layer (type)                 Output Shape              Param \#   
=================================================================
dense\_1 (Dense)              (None, 512)               401920    
\_\_\_\_\_\_\_\_\_\_\_\_\_\_\_\_\_\_\_\_\_\_\_\_\_\_\_\_\_\_\_\_\_\_\_\_\_\_\_\_\_\_\_\_\_\_\_\_\_\_\_\_\_\_\_\_\_\_\_\_\_\_\_\_\_
dropout\_1 (Dropout)          (None, 512)               0         
\_\_\_\_\_\_\_\_\_\_\_\_\_\_\_\_\_\_\_\_\_\_\_\_\_\_\_\_\_\_\_\_\_\_\_\_\_\_\_\_\_\_\_\_\_\_\_\_\_\_\_\_\_\_\_\_\_\_\_\_\_\_\_\_\_
dense\_2 (Dense)              (None, 512)               262656    
\_\_\_\_\_\_\_\_\_\_\_\_\_\_\_\_\_\_\_\_\_\_\_\_\_\_\_\_\_\_\_\_\_\_\_\_\_\_\_\_\_\_\_\_\_\_\_\_\_\_\_\_\_\_\_\_\_\_\_\_\_\_\_\_\_
dropout\_2 (Dropout)          (None, 512)               0         
\_\_\_\_\_\_\_\_\_\_\_\_\_\_\_\_\_\_\_\_\_\_\_\_\_\_\_\_\_\_\_\_\_\_\_\_\_\_\_\_\_\_\_\_\_\_\_\_\_\_\_\_\_\_\_\_\_\_\_\_\_\_\_\_\_
dense\_3 (Dense)              (None, 10)                5130      
=================================================================
Total params: 669,706
Trainable params: 669,706
Non-trainable params: 0
\_\_\_\_\_\_\_\_\_\_\_\_\_\_\_\_\_\_\_\_\_\_\_\_\_\_\_\_\_\_\_\_\_\_\_\_\_\_\_\_\_\_\_\_\_\_\_\_\_\_\_\_\_\_\_\_\_\_\_\_\_\_\_\_\_

    \end{Verbatim}

    \begin{Verbatim}[commandchars=\\\{\}]
{\color{incolor}In [{\color{incolor} }]:} \PY{n}{model}\PY{o}{.}\PY{n}{compile}\PY{p}{(}\PY{n}{loss}\PY{o}{=}\PY{l+s+s1}{\PYZsq{}}\PY{l+s+s1}{categorical\PYZus{}crossentropy}\PY{l+s+s1}{\PYZsq{}}\PY{p}{,}
                      \PY{n}{optimizer}\PY{o}{=}\PY{n}{RMSprop}\PY{p}{(}\PY{p}{)}\PY{p}{,}
                      \PY{n}{metrics}\PY{o}{=}\PY{p}{[}\PY{l+s+s1}{\PYZsq{}}\PY{l+s+s1}{accuracy}\PY{l+s+s1}{\PYZsq{}}\PY{p}{]}\PY{p}{)}
\end{Verbatim}


    \begin{Verbatim}[commandchars=\\\{\}]
{\color{incolor}In [{\color{incolor}28}]:} \PY{n}{batch\PYZus{}size} \PY{o}{=} \PY{l+m+mi}{12}
         \PY{n}{epochs} \PY{o}{=} \PY{l+m+mi}{20}
         \PY{n}{history} \PY{o}{=} \PY{n}{model}\PY{o}{.}\PY{n}{fit}\PY{p}{(}\PY{n}{x\PYZus{}train}\PY{p}{,} \PY{n}{y\PYZus{}train}\PY{p}{,}
                             \PY{n}{batch\PYZus{}size}\PY{o}{=}\PY{n}{batch\PYZus{}size}\PY{p}{,}
                             \PY{n}{epochs}\PY{o}{=}\PY{n}{epochs}\PY{p}{,}
                             \PY{n}{verbose}\PY{o}{=}\PY{l+m+mi}{0}\PY{p}{,}
                             \PY{n}{validation\PYZus{}data}\PY{o}{=}\PY{p}{(}\PY{n}{x\PYZus{}test}\PY{p}{,} \PY{n}{y\PYZus{}test}\PY{p}{)}\PY{p}{)}
\end{Verbatim}


    \begin{Verbatim}[commandchars=\\\{\}]
{\color{incolor}In [{\color{incolor} }]:} \PY{n}{score} \PY{o}{=} \PY{n}{model}\PY{o}{.}\PY{n}{evaluate}\PY{p}{(}\PY{n}{x\PYZus{}test}\PY{p}{,} \PY{n}{y\PYZus{}test}\PY{p}{,} \PY{n}{verbose}\PY{o}{=}\PY{l+m+mi}{0}\PY{p}{)}
        \PY{n+nb}{print}\PY{p}{(}\PY{l+s+s1}{\PYZsq{}}\PY{l+s+s1}{Test loss}\PY{l+s+s1}{\PYZsq{}}\PY{p}{,} \PY{n}{score}\PY{p}{[}\PY{l+m+mi}{0}\PY{p}{]}\PY{p}{)}
        \PY{n+nb}{print}\PY{p}{(}\PY{l+s+s1}{\PYZsq{}}\PY{l+s+s1}{Test accuracy}\PY{l+s+s1}{\PYZsq{}}\PY{p}{,} \PY{n}{score}\PY{p}{[}\PY{l+m+mi}{1}\PY{p}{]}\PY{p}{)}
\end{Verbatim}


    \subsection{化学での応用}\label{ux5316ux5b66ux3067ux306eux5fdcux7528}


    % Add a bibliography block to the postdoc
    
    
    
    \end{document}
